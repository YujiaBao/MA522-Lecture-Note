\documentclass[12pt]{article}
\usepackage{geometry}
\geometry{left=2cm,right=2cm,top=2cm,bottom=2.5cm}
\usepackage{amsmath}
\DeclareMathOperator{\image}{Im}
\DeclareMathOperator{\kernal}{Ker}
\DeclareMathOperator{\Aut}{Aut}
\DeclareMathOperator{\Inv}{Inv}
\DeclareMathOperator{\interior}{interior}
\DeclareMathOperator{\Gal}{Gal}
\DeclareMathOperator{\subgp}{subgp}
\DeclareMathOperator{\subfield}{subfield}
\usepackage{amssymb}
\usepackage{mathtools}
\usepackage{enumitem}
\usepackage{mathrsfs}
%\usepackage{mathptmx}
%\usepackage{times}
\usepackage{setspace}
\usepackage{extarrows}
\usepackage{amsthm}
\usepackage{bm}
\usepackage{float}
\usepackage{tikz}
\usepackage{fancybox}
\theoremstyle{plain}
\newtheorem{thm}{Theorem}
\newtheorem*{lemma}{Lemma}
\newtheorem*{prop}{Proposition}
\newtheorem*{corollary}{Corollary}
\theoremstyle{definition}
\newtheorem*{defn}{Definition}
\newtheorem*{eg}{Example}
\newtheorem*{ex}{Exercise}
\newtheorem*{Remarks}{Remarks}
\newtheorem*{note}{Note}
\newtheorem*{recall}{Recall}
\newtheorem*{property}{Properties}

\begin{document}

\title{MA522 Lecture Notes}
\author{Instructor: Betsy Stovall\qquad Note taken by: Yujia Bao}
\date{}
\maketitle
% Week 1, Wednesday
\noindent\rule{\textwidth}{1pt}
\begin{defn}
    Let $(X,d)$ be a metric space, let $K\subseteq X$, then
    \begin{itemize}
        \item
            An open cover of $K$ (in $X$) is a set
            $\mathcal{G}=\{G_\alpha\}_{\alpha\in \mathcal{A}}$
        where each $G_\alpha$ is an open subset of $X$ and
        $K\subseteq\cup_{\alpha\in \mathcal{A}}G_{\alpha}$.
        \item
            $K$ is compact if every open cover of $K$ contains a finite subcover
            of $K$, i.e.\ if for every open cover
            $\mathcal{G}=\{G_\alpha\}_{\alpha\in \mathcal{A}}$, there $\exists\,
            \alpha_1,\ldots,\alpha_N\in \mathcal{A}$, s.t. $K\subseteq\cup_{j=1}^N
            G_{\alpha_j}$.
    \end{itemize}
\end{defn}
\begin{eg}
    Show that $(0,1]$ is not compact.
\end{eg}
\begin{proof}
    Let $\mathcal{G}=\{(1/n,\,2):\,n\in\mathbb{N}\}$.
    Then $\mathcal{G}$ is an open cover of $(0,1]$.
    But if $\{n_1,\ldots,n_N\}$ is any finite set,
    $\cup_{j=1}^N(1/n,\,2)=(1/\max n_j,\,2)\nsubseteq (0,1]$.
\end{proof}

\begin{eg}
    Show that $\mathbb{R}$ is not compact.
\end{eg}
\begin{proof}
    Let $\mathcal{G}=\{(-n,n):\,n\in\mathbb{N}\}$.
\end{proof}
\begin{thm}[Heine-Borel]\label{thm:HB}
    A subset of $\mathbb{R}^n$ is compact if and only if it is closed and
    bounded.
\end{thm}


\begin{eg}
    Show that $\{0\}\cup\{1/n:\,n\in\mathbb{N}\}$ is compact.
\end{eg}
\begin{proof}
    Let $\mathcal{G}=\{G_{\alpha}\}_{\alpha\in \mathcal{A}}$ be any open cover.
    Then $0\in G_{\alpha_0}$ for some $\alpha_0\in \mathcal{A}$.
    Since $G_{\alpha_0}$ is open, $\exists \epsilon>0$ such that
    $B_{\epsilon}(0)\subseteq G_{\alpha_0}$.
    For $N=\lceil\epsilon\rceil$, we have $1/n\in B_{\epsilon}(0)\subseteq
    G_{\alpha_0}$ for all $n>N$.
    Choose $\alpha_n$ such that $1/n\in G_{\alpha_n}$ for each $n\leq N$.
    Thus $\{G_{\alpha_j}\}_{j=0}^N$ is a finite subcover of the origin set.
    So the origin set is compact.
\end{proof}

\begin{defn}
    $U\subseteq X$ is precompact if $\bar{U}$ is compact. (Here $\bar{U}$ stands
    for the closure of $U$.)
\end{defn}

\begin{eg}
    By Theorem~\ref{thm:HB}, every Borel subset of $\mathbb{R}^n$ is precompact.
\end{eg}
%Question: Does it mean for every subset of $\mathbb{R}^n$, as long as it is
%bounded, it is precompact?
Question: Definition of Borel set.

\begin{defn}
    $K\subseteq X$ is sequentially compact if every sequence $\{x_n\}$ contains
    a subsequence $\{x_{n_k}\}$ that converges to a limit in $K$.
\end{defn}

\begin{defn}
    $U\subseteq{X}$ is totally bounded if $\forall \epsilon>0$, $U$ is covered
    by a finite collection of $\epsilon$-balls, i.e.,
    $\exists\,x_1,\ldots,x_{N_\epsilon}\in X$, s.t.
    $U\subseteq\cup_{j=1}^{N_\epsilon}
    B_{\epsilon}(x_j)$.
\end{defn}
\begin{eg}
    Every bounded subset of $\mathbb{R}^n$ is totally bounded.
\end{eg}
\begin{eg}
    In discrete metric space,
    $$\delta(x,y)=
    \left\{
        \begin{aligned}
            1\quad&\text{if }x\neq y,\\
            0\quad&\text{if }x= y.\\
        \end{aligned}
    \right.$$
    Then, every set is both open and closed.
    For $\epsilon<1$, the $\epsilon$ Ball becomes a single ball.
    Infinite sets are bounded but not totally bounded.
\end{eg}

\begin{defn}
    $K\subseteq X$ is complete if every Cauchy sequence in $K$ converges to
    some limit in $K$.
\end{defn}

\begin{defn}
    $x$ is an accumulation point of $E\subseteq X$ if $\forall \epsilon>0$,
    $(B_{\epsilon}(x)\setminus\{x\})\cap E\neq \emptyset$.
    This is equivalent to say $\forall \epsilon > 0, B_{\epsilon}(x)\cap E$
    contains infinitly many points.
\end{defn}

\begin{thm}
    Let $(X,d)$ be a metric space, then TFAE
    \begin{enumerate}
        \item
            $K$ is compact.
        \item
            $K$ has the Bolzano-Weierstrass property
            (Every infinity subset of $K$ has an accumulation point in $K$).
        \item
            $K$ is sequentially compact.
        \item
            $K$ is complete and totally bounded.
    \end{enumerate}
\end{thm}

\begin{proof}
    \begin{description}
        \item[$1\Rightarrow 2$:]
            Assume $K$ is compact.
            Let $E\subseteq K$ be an infinite set.
            If there doesn't exist such $E$, then Bolzano-Weierstrass property
            holds trivially.
            Now suppose $E$ has no accumulation point in $K$.
            That means for every $x$ in $K$, $\exists$ neighbour $U_x$ of $x$
            (i.e. $x\in U_x$ and $U_x$ is open), that contains no points of $E$
            other than (possibly) $x$ itself.
            Since $K=\cup_{x\in K}\{x\}\subseteq\cup_{x\in K} U_x$, $\{U_x:\,x\in
            K\}$ is an open cover of $K$.
            By compactness, $\exists x_1,\ldots,x_N$, s.t. $K\subseteq\cup_{j=1}^N
            U_{x_j}$.
            But $\cup_{j=1}^N U_{x_j}$ contains at most $N$ points of $E$.
            So it cannot contain all points of $E$ because $E$ is an infinity
            set.
            Contradiction! Since $E\subseteq K$, $\cup_{j=1}^N U_{x_j}$ should
            be a cover of $E$.
        \item[$2\Rightarrow 3$:]
            Assume $K$ has the Bolzano-Weierstrass property.
            Let $\{x_n\}$ be a sequence in $K$. We need to show that it has a
            convergent subsequence.
            Let $E=\{x_n:\,n\in\mathbb{N}\}$.
            \begin{description}
                \item[Case I:]
                    $E$ is a finite set.
                    By the pigeonhole principle, $\exists x\in E\subseteq K$,
                    s.t. $x_n=x$ for infinitely many $n$.
                    Here $\{x_n\}$ has a constant subsequence which only takes
                    the value $x$.
                    This is a subsequnce converge to $x\in K$.
                \item[Case II:]
                    $E$ is an infinite set.
                    By Bolazano-Weierstrass propert, $E$ has an accumulation
                    point $x\in K$.
                    Thus every ball centered at $x$ contains infinitely many
                    $x_n$s.
                    Choose $n_1$ such that $x_{n_1}\in B_1(x)$;
                    Choose $n_2>n_1$ such that $x_{n_2}\in B_{1/2}(x)$; and so
                    on.
                    Proceeding by induction, we may find
                    $n_1<n_2<\cdots<n_k<\cdots$, s.t. $x_{n_k}\in B_{1/k}(x)$
                    for all $k$.
                    Then $\{x_{n_k}\}$ is a subsequent of $\{x_n\}$ and
                    $x_{n_k}\rightarrow x$.
            \end{description}
        \item[$3\Rightarrow 4$:]
            Assume $K$ is sequentially compact.
            \begin{description}
                \item[Completeness:]
                    Let $\{x_n\}$ be a Cauchy sequence in $K$.
                    By sequentially compactness, there exists a convergent
                    subsequent $\{x_{n_k}\}$ of $\{x_n\}$, say
                    $x_{n_k}\rightarrow x\in K$.
                    Then we claim that $x_n\rightarrow x$.
                    Since $x_{n_k}\rightarrow x$,
                    let $\epsilon>0$, $\exists M$ s.t. $\forall k\geq M$,
                    $d(x_{n_k},x)<\epsilon$.
                    Since $\{x_n\}$ is a Cauchy sequence, $\exists N$ s.t.
                    $\forall n,m\geq N$, $d(x_n,x_m)<\epsilon$.
                    Now fix $k_0\geq \max\{M,N\}$ and let $n\geq N$.
                    Then
                    $$d(x_n,x)\leq d(x_n,x_{n_{k_0}})+d(x_{n_{k_0}},x).$$
                    Since $k_0\geq M$, we have $d(x_{n_{k_0}},x)<\epsilon$.
                    Since $n_{k_0}\geq k_0\geq N$, we have $d(x_n,x_{n_{k_0}})<\epsilon$.
                    Then $d(x_n,x)<2\epsilon$. So $\{x_n\}$ does converge.
                \item[Totally boundedness:]
                    Suppose not.
                    Then $\exists \epsilon>0$ s.t. $K$ cannot be covered by a
                    finite union of $\epsilon$-balls.
                    Thus, we may (inductively) construct a sequence $\{x_n\}$ in
                    $K$ such that $\forall n\geq 2$,
                    $x_n\notin\cup_{j=1}^{n-1}B_\epsilon(x_j)$.
                    Let $\{x_{n_k}\}$ be any subsequence of $\{x_n\}$.
                    Pick any $k_1, k_2$ with $k_1<k_2$.
                    Then $x_{k_2}\notin B_\epsilon(x_{k_1})$ which means
                    $d(x_{n_{k_1}},x_{n_{k_2}})\geq \epsilon$.
                    So $\{x_{n_k}\}$  is not Cauchy and hence not convergent.
                    This contradicts with sequential compactness of $K$.
            \end{description}
        \item[$4\Rightarrow 3$:]
            Assume $K$ is complete and totally bounded.
            Let $\{x_n\}$ be a sequence in $K$.
            We want to find a convergent subsequence.
            By totally boundedness, $K$ is covered by a finite number of
            $1$-balls, $K\subseteq\cup_{j=1}^N B_1(y_j)$.
            By the pigenhole principle, there must exist an $B_1(y_j)$ which
            contains $x_n$ for infinitely many $n$.
            Denote that $y_j$ as $z_1$.
            So there exists a subsequence $\{x_{n_k^1}\}$ contained in
            $B_1(z_1)$.
            By the same argument, $\exists z_2$ s.t. $B_{1/2}(z_2)$ contains a
            subsequent $\{x_{n_k^2}\}$ of $\{x_{n_k^1}\}$, and so on.
            So for each $m\in\mathbb{N}$, we find $z_m$ and a subsequent
            $\{x_{n_k^m}\}$ of $\{x_{n_k^{m-1}}\}$ s.t. $\{x_{n_k^m}\}$ is
            contained in $B_{1/m}(z_m)$.
            Now we define $x_{n_k}=x_{n_k^k}$ (diagonalization).
            \begin{description}
                \item [Claim 1:]
                    $\{x_{n_k}\}$ is a subsequence of $\{x_n\}$.\\
                    This is because $n_k=n_k^k\geq
                    n_k^{k-1}>n_{k-1}^{k-1}=n_{k-1}$, where the first inequality
                    comes from the fact that $\{x_{n_j^k}\}$ is a subsequence of
                    $\{x_{n_j^{k-1}}\}$.
                \item [Claim 2:]
                    $\{x_{n_k}\}$ is a Cauchy sequence.\\
                    For $k_1, k_2\geq M$, $x_{n_{k_1}}$ and $x_{n_{k_2}}$ are
                    both terms in the sequence $\{x_{n_j^M}\}$.
                    So both of them lie in $B_{1/M}(z_M)$ which means
                    $d(x_{n_{k_1}},x_{n_{k_2}})\leq 2/M$.
            \end{description}
            By completeness of $K$, $\{x_{n_k}\}$ converges in $K$.
        \item[$4\Rightarrow 2$:]
            Assume $K$ is complete and totally bounded.
            Let $E\subseteq K$ be an infinite subset.
            Since $K$ is totally bounded, $K$ can be covered by finitely many
            $1$-balls.
            By pigeonhole principle, $\exists x_1\in K$, s.t.
            $B_1(x_1)\cap E\eqqcolon E_1$ is an infinite set.
            By induction, for each $n\in\mathbb{N}^+$, $\exists x_n\in K$, s.t.
            $B_{1/n}(x_n)\cap E_{n-1}\eqqcolon E_n$ is an infinite set.
            \begin{description}
                \item[Claim 1: ]$\{x_n\}$ is a Cauchy sequence.\\
                    Notice that $\forall n,m$,
                    $$B_{1/n}(x_n)\cap B_{1/m}(x_m)\supseteq
                    E_{\max\{m,n\}}\neq\emptyset.$$
                    This implies
                    $$d(x_n,x_m)<\frac{1}{n}+\frac{1}{m}\leq\frac{2}{\min\{n,m\}}.$$
                    Then as long as $n,m>2/\epsilon$, we have
                    $d(x_n,x_m)<\epsilon$.
                    So $\{x_n\}$ is a Cauchy sequence.
            \end{description}
            By completeness of $K$, $\{x_n\}$ converges, say $x_n\rightarrow
            x_0$.
            \begin{description}
                \item[Claim 2: ]$x_0$ is an accumulation point for $E$.\\
                    Let $\epsilon>0$. Choose $N$ sufficiently large, such
                    that $\forall n\geq N$, $d(x_0,x_n)<\epsilon/2$ and
                    $1/n<\epsilon/2$.
                    For any $y\in B_{1/n}(x_n)$, we have $d(y,x_n)<1/n$.
                    Then
                    $$d(x_0,y)\leq d(x_0,x_n)+d(x_n,y)
                    <\frac{\epsilon}{2}+\frac{1}{n}
                    <\frac{\epsilon}{2}+\frac{\epsilon}{2}=\epsilon$$
                    So we have $y\in B_{\epsilon}(x_0)$.
                    Then $B_{\epsilon}(x_0)\supseteq B_{1/n}(x_n)\supseteq E_n$.
                    Since $E_n$ is an infinite subset of $E$ and $\epsilon$ is
                    selected arbitrary, $x_0$ is an accumulation point of $E$.
            \end{description}
        \item[$4, 3\Rightarrow 1$:]
            Assume $K$ is complete, totally bounded and sequentially compact.
            Let $\mathcal{G}\coloneqq\{G_\alpha\}_{\alpha\in\mathcal{A}}$ be an
            open cover of $K$.
            Then $\forall x\in K$, $\exists \alpha\in\mathcal{A}$, s.t. $x\in
            G_\alpha$.
            Since $G_\alpha$ is open, $\exists r>0$, s.t. $B_r(x)\subseteq
            G_\alpha$.
            Thus we define
            $$\epsilon(x)\coloneqq \sup\{r>0\,:\,\exists\alpha\in\mathcal{A},
            \text{ s.t. } B_r(x)\subseteq G_\alpha\}$$
            By definition , $\epsilon(x)>0$.
            If $\epsilon(x)=+\infty$ for some $x\in K$, then since $K$ is
            bounded (because $K$ is totally bounded), there must be one
            $G_\alpha$ containing all of $K$.
            Now we assume $\epsilon(x)$ is finite for all $x\in K$.
            \begin{description}
                \item[Claim: ]$\exists \epsilon_0>0$, s.t. $\epsilon(x)\geq
                    \epsilon_0$, $\forall x\in K$.\\
                    Suppose such $\epsilon_0$ doesn't exist.
                    Then $\exists\{x_n\}$, s.t. $\epsilon(x_n)\rightarrow 0$.
                    By sequential compactness of $K$, there exists a subsequence
                    $x_{n_k}\rightarrow x_0\in K$.
                    As $\epsilon(x_0)>0$, we can choose some $r>0$ and $\alpha$
                    s.t. $B_r(x_0)\subseteq G_\alpha$.
                    But for $k$ sufficiently large,
                    $\epsilon(x_{n_k})<r/2$ and $d(x_0,x_{n_k})<r/2$.
                    Then
                    $$B_{r/2}(x_{n_k})\subseteq B_r(x_0)\subseteq G_\alpha,$$
                    and this implies $r/2\leq\epsilon(x_{n_k})<r/2$ which leads
                    to contradiction.
            \end{description}
            Since $K$ is totally bounded, so $\exists x_1,\ldots,x_N$, s.t.
            $$K\subseteq \bigcup\limits_{j=1}^N B_{\epsilon_0/2}(x_j)
            \subseteq \bigcup_{j=1}^N B_{\epsilon(x_j)/2}(x_j)$$
            Furthermore, as $\epsilon(x_j)/2<\epsilon(x_j)$ (because
            $\epsilon(x_j)>0$), $\exists r>0$, s.t. $r>\epsilon(x_j)/2$ and
            $B_r(x_j)\subseteq G_{\alpha_j}$ for some $\alpha_j\in\mathcal{A}$.
            Finally, $\cup_{j=1}^N G_{\alpha_j}$ is a finite subcover of $K$.
            Thus $K$ is compact.
    \end{description}
\end{proof}

\begin{corollary}
    A subset of a complete metric space is compact if and only if it is closed
    and totally bounded.
\end{corollary}

\begin{corollary}
    A subset of a complete metric space is precompact if and only if it is
    totally bounded.
\end{corollary}

\begin{eg}
    A closed subset of a compact set is compact.
    \begin{proof}
        Suppose $K\subseteq E$. Since $E$ is compact, by Bolzano-Weierstrass
        property, every infinite subset of $K$ has an accumulation point.
        Since $K$ is closed, $K$ contains all its accumulation points.
        So $K$ also satisfies B-W property which means $K$ is compact.
    \end{proof}
\end{eg}

\begin{defn}
    A subset $E$ of a metric space $X$ is not connected if $\exists$ sets $A, B$
    s.t.
    $$\left\{
        \begin{aligned}
            &E=A\cup B,\ A\neq\emptyset,\ B\neq\emptyset\\
            &\bar{A}\cap B=\emptyset,\ A\cap\bar{B}=\emptyset\\
        \end{aligned}
    \right.$$
    We say $A$ and $B$ form a separation of $E$, or $A$ and $B$ separate $E$.
\end{defn}

\begin{defn}
    $E$ is connected if it is not disconnected.
\end{defn}

\begin{thm}\label{thm:3}
    Let $E$ be a subset of $X$. Then TFAE:
    \begin{enumerate}
        \item $E\subseteq X$ is not connected.
        \item $\exists$ open sets $U,V\subseteq X$, s.t.
            $$E\subseteq U\cup V,\ E\cap U\neq\emptyset,\ E\cap V\neq\emptyset,\
            U\cap V=\emptyset.$$
        \item $\exists A\subseteq E$, s.t.
            $$A\neq\emptyset,\ A\neq E,\ A=E\cap F=E\cap G,$$
            where $F$ is closed and $G$ is open.
    \end{enumerate}
\end{thm}

\begin{proof}
    \begin{description}
        \item[$2\Rightarrow 3$:]
            Assume $2$ holds.
            We pick $A=E\cap U$.
            Then $A\neq\emptyset$.
            Let $G=U$. Then $A=E\cap G$.
            Let $F=V^c$. Then $F$ is closed since $V$ is open.
            \begin{description}
                \item[Claim 1:]
                    $A=E\cap F$.\\
                    Let $a\in A$.
                    Then $a\in E$ and $a\in U\subseteq V^c=F$.
                    So $a\in E\cap F$.
                    Let $x\in E\cap F$.
                    Then $x\in E$ and $x\notin V$.
                    Since $E\subseteq U\cup V$, $x\in U$.
                    So $x\in E\cap U= A$.
                \item[Claim 2:]
                    $A\neq E$.\\
                    This is true because $E\setminus A=E\setminus(E\cap F)=E\cap
                    V$ and  we know $E\cap V\neq\emptyset$ by 2.
            \end{description}
        \item[$3\Rightarrow 1$:]
            Assume 3 holds.
            We define $A$ as in 3 and let $B=E\setminus A$.
            Then by definition, $A\cup B=E$, $A\neq \emptyset$ and $B\neq \emptyset$
            (since $A\neq E$).
            Note that
            $$\bar{A}\cap B=\overline{E\cap F}\cap(E\cap F^c)\subseteq \bar{F}\cap
            F^c=F\cap F^c=\emptyset,$$
            since $F$ is closed.
            Also
            $$A\cap\bar{B}=E\cap G\cap\overline{E\cap G^c}\subseteq
            G\cap\overline{G^c}=G\cap G^c=\emptyset,$$
            since $G^c$ is closed.
            Then $E$ is not connected.
        \item[$1\Rightarrow 2$:]
            Assume 1 holds.
            Let $a\in A$.
            Then $a\notin\bar{B}$, since $A\cap\bar{B}=\emptyset$.
            So $\exists r(a)>0$, s.t. $B_{r(a)}(a)\cap B=\emptyset$.
            Likewise, if $b\in B$, $\exists r(b)>0$, s.t. $B_{r(b)}(b)\cap
            A=\emptyset$.
            Now define $U=\cup_{a\in A}B_{r(a)/2}(a)$ and 
            $V=\cup_{b\in B}B_{r(b)/2}(b)$.
            Then $U$ and $V$ are open sets since they are unions of open sets.
            Also $E\subseteq A\cup B\subseteq U\cup V$ and
            $E\cap U\supseteq A\neq\emptyset$, $E\cap V\supseteq
            B\neq\emptyset$.
            Now we just need to show $U\cap V=\emptyset$.
            If it is not true, $\exists x\in U\cap V$.
            By our construction of $U$ and $V$, $\exists a\in A,\,b\in B$, s.t.
            $x\in B_{r(a)/2}(a)\cap B_{r(b)/2}(b)$.
            So
            $$d(a,b)\leq d(a,x)+d(b,x)<\frac{r(a)}{2}+\frac{r(b)}{2}\leq
            \max\{r(a),\,r(b)\}$$
            Then if $r(a)\geq r(b)$, we have $b\in B_{r(a)}(a)$.
            If $r(b)\geq r(a)$, we have $a\in B_{r(b)}(b)$.
            Both of them contradicts with our definition of $r(a)$ or $r(b)$.
            So $U\cap V=\emptyset$.
    \end{description}
\end{proof}

\begin{eg}
    $(-\infty,\,0)\cup(0,\,+\infty)$ is not connected.
\end{eg}

\begin{eg}
    $E\coloneqq\{(x,y)\,:\,y\in[-1,1]\text{ with }x=0\text{ or
    }y=\sin(1/x)\text{ with }x>0\}$ is connected.
\end{eg}

\begin{prop}
    If $E$ is connected and $f:\,E\rightarrow Y$ is continuous,
    then $f(E)$ is connected.
\end{prop}
\begin{proof}
    Assume $f(E)$ is not connected.
    Then $\exists U,V$ open with $f(E)\subseteq U\cup V,\,U\cap
    V=\emptyset,\,f(E)\cap U\neq\emptyset,\,f(E)\cap V\neq\emptyset$.
    Note that $f^{-1}(U)$ is open, nonempty.
    Likeliwise for $f^{-1}(V)$.
    Moreover $f^{-1}(U)\cap f^{-1}(V)=\emptyset$ (If $\exists x\in f^{-1}(U)\cap
    f^{-1}(V)$, then $f(x)\in U\cap V=\emptyset$. Contradiction!)
    and $f^{-1}(U)\cup f^{-1}(V)\supseteq E$.
    By Theorem~\ref{thm:3}, $E$ is not connected.
    Contradiction!
\end{proof}

%The Contraction Mapping Principle
\begin{defn}
    Let $(X,D)$ be a metric space. (Assume $X\neq\emptyset$.)
    Say $\Phi:\,X\rightarrow X$ is a contraction if $\exists r<1$, s.t.
    $\forall x,y\in X$, $$d(\Phi(x),\,\Phi(y))\leq r\cdot d(x,y)$$
\end{defn}

\begin{eg}
    By mean value theorem, if $f:\,\mathbb{R}\rightarrow \mathbb{R}$ is
    differentiable and $\exists r<1$, s.t. $\forall x$, $|f'(x)|\leq r$.
    Then $f$ is a contraction.
\end{eg}

\begin{prop}
    If $\Phi$ is a contraction, then $\Phi$ is continuous.
\end{prop}

\begin{thm}[Contraction Mapping Theorem]
    If $(X,d)$ is a nonempty complete metric space and $\Phi:\,X\rightarrow X$ is a
    contraction.
    Then $\Phi$ has a unique fixed point, i.e. $\exists! x_0\in X$, s.t.
    $\Phi(x_0)=x_0$.
\end{thm}
\begin{proof}
    Let $\Phi$ be a contraction with shrinking constant $r<1$.
    \begin{description}
        \item[Uniqueness:]
            Suppose $x_1, x_2$ are both fixed points of the contraction $\Phi$.
            Then
            $$d(x_1,x_2)=d(\Phi(x_1),\Phi(x_2))\leq r\cdot d(x_1,x_2).$$
            Thus $d(x_1,x_2)=0$ and $x_1=x_2$, since $r<1$.
        \item[Existence:]
            Let $x\in X$.
            Define a sequence $\{x_n\}$ inductively by setting $x_0=x$ and
            $x_n=\Phi(x_{n-1})$ for $n\in\mathbb{N}^+$.
            \begin{description}
                \item[Claim 1:]
                    $\{x_n\}$ is a Cauchy sequence and hence convergent.\\
                    Suppose $m,n\in\mathbb{N}^+$, $m>n$.
                    Then
                    $$d(x_m,x_n)=d(\Phi(x_{m-1}),\Phi(x_{n-1}))\leq
                    rd(x_{m-1},x_{n-1}).$$
                    By induction,
                    $$\begin{aligned}
                        d(x_m,x_n)&\leq r^n d(x_{m-n},x_0)\\
                        &\leq r^n\left(d(x_{m-n},x_{m-n-1})+d(x_{m-n-1},x_{m-n-2})
                        +\cdots+d(x_{1},x_{0})\right)\\
                        &\leq r^n\left(r^{m-n-1}d(x_1,x_0)+ r^{m-n-2}d(x_1,x_0)+
                        \cdots+d(x_1,x_0)\right)\\
                        &\leq r^n d(x_1,x_0) \sum_{j=0}^\infty r^j\\
                        &=\frac{r^n}{1-r}d(x_1,x_0)
                    \end{aligned}$$
                    Given any $\epsilon>0$, choose $N$ such that
                    $r^Nd(x_1,x_0)/(1-r)<\epsilon$.
                    Then by the preceeding calculation, $\forall n,m\geq N$,
                    $d(x_n,x_m)<\epsilon$.
                \item[Claim 2:]
                    $\lim_{n\rightarrow\infty}x_n$ is a fixed point of $\Phi$.\\
                    Since $\Phi$ is a contraction, $\Phi$ is continuous. So we
                    can exchange $\Phi$ with the limit operation. Then
                    $$\Phi(\lim_{n\rightarrow \infty} x_n)=
                    \lim_{n\rightarrow \infty}\Phi(x_n)
                    =\lim_{n\rightarrow\infty}x_{n+1}
                    =\lim_{n\rightarrow\infty}x_{n}.$$
            \end{description}
    \end{description}
\end{proof}

\begin{corollary}
    If $\exists r<1$, s.t. $|f'(x)|\leq r$, $\forall x\in\mathbb{R}$.
    Then $\exists! x_0$, s.t. $f(x_0)=x_0$.
\end{corollary}

\begin{eg}
    $\Phi(x)=x+1$ on $\mathbb{R}$ is not a contraction.
\end{eg}

\begin{eg}
    $X=(-\infty,0)\cup(0,\infty)$ is not complete.
    For $\Phi(x)=\frac{1}{2}x$, it is a contraction but the fixed point $x_0=0$
    is not in $X$.
\end{eg}

%Bare Category Theorem
\begin{defn}
    Let $E\subseteq X$.
    \begin{itemize}
        \item
            $E$ is dense in $X$ if $\bar{E}=X$.
        \item
            $E$ is nowhere dense in $X$ if $\bar{E}$ has empty interior.
    \end{itemize}
\end{defn}

\begin{eg}
    $\mathbb{Z}\subseteq \mathbb{R}$ is nowhere dense in $\mathbb{R}$.
    $\mathbb{Q}\cap (-2,2)$ is neither dense nor nowhere dense in $\mathbb{R}$.
\end{eg}

\begin{defn}
    Let $E\subseteq X$, where $X$ is some metric space.
    \begin{itemize}
        \item
        $E$ is meager if $E$ can be written as a countable union of nowhere
        dense sets.
        \item
        $E$ is generic if $E^c$ is meager.
    \end{itemize}
\end{defn}

\begin{thm}[Baire Category Theorem]\label{thm:5}
    A nonempty complete metric space cannot be written as a countable union of nowhere
    dense sets.
\end{thm}
    This is equivalent to say that a complete metric space cannot be meager and
    is also equivalent to say that a subset of a complete metric space cannot be
    both meager and generic.
    In particular, generic subsets of complete metric space are nonempty.
\begin{proof}
    Let $(X,d)$ be a complete metric space and $\{F_n\}$ be a collection of
    nowhere dense subsets of $X$.
    Suppose for contradiction that $X=\cup_nF_n$.
    Then $X=\cup_n\bar{F}_n$ and $\bar{F}_n$ are also nowhere dense.
    Without loss of generality, we assume each $F_n$ is closed.\\
    Since $X\nsubseteq F_i$ for any $i$ (Otherwise, $\overline{\interior
    (F_i)}=X\neq\emptyset$.),
    $\exists x_1\in F_1^c$ and $\exists r_1>0$, s.t.
    $\bar{B}_{r_1}(x_1)\subseteq F_1^c$.
    Since $\interior(F_2)=\emptyset$, $B_{r_1}(x_1)\nsubseteq F_2$.
    Then $\exists x_2$, s.t. $x_2\in F_2^c\cap B_{r_1}(x_1)$.
    Since $F_2^c$ and $B_{r_1}(x_1)$ are both open, their intersection is open.
    $\exists r_2>0$, s.t. $\bar{B}_{r_2}(x_2)\subseteq F_2^c\cap B_{r_1}(x_1)$ and
    $r_2\leq r_1/2$.
    By induction, we obtain a sequence $\{x_n\}$ in $X$ and $\{r_n\}$ in
    $(0,+\infty)$, s.t.\ for all $n$,
    $$B_{r_n}(x_n)\subseteq F_n^c\cap B_{r_{n-1}}(x_{n-1}),\quad
    r_n\leq\frac{1}{2}r_{n-1}.$$
    So $r_n\rightarrow0$.
    \begin{description}
        \item[Claim 1:]
            $\{x_n\}$ is a Cauchy sequence.\\
            For $n,m\geq M$, $x_n, x_m\in B_{r_M}(x_M)$.
            So $d(x_n,x_m)<r_M$ and $r_M\rightarrow0$.
        \item[Claim 2:]
            $x_\infty\coloneqq \lim_{x\rightarrow\infty}x_n\notin\cup_n F_n$.\\
            Since $\{x_n\}_{n\geq M}$ is a sequence in $B_{r_M}(x_M)$,
            $x_\infty\in\overline{B_{r_M}(x_M)}\subseteq F_M^c$.
            As $M$ was arbitrary,
            $$x_\infty\in\bigcap_M F_M^c=\left(\bigcup_M F_M\right)^c.$$
    \end{description}
    Then we have $x_\infty\notin \cup_{n}F_n=X$.
    However, $X$ is a complete metric. Contradiction.
\end{proof}

\begin{prop}
    $\mathbb{R}^n$ cannot be written as a countable union of hyperplanes.
\end{prop}

\begin{proof}
    Let $P$ be any hyperplane. So
    $P=\{x\in\mathbb{R}^n\,:\,\langle x,a\rangle=d\}$ for fixed $a, d$, where
    $a\neq 0$.
    \begin{description}
        \item[Claim 1:]
            Hyperplane $P$ is closed.\\
            $P$ can also be defined by $f^{-1}(d)$ where $f(x)=\langle
            x,a\rangle$.
            Since $d$ is closed, $f$ is continuous, the preimage $f^{-1}$ of a
            closed set $d$ is also closed.
        \item[Claim 2:]
            Hyperplane $P$ has empty interior.\\
            Let $x_0\in P$.
            For any $r>0$,
            the point $x_0+(r/2|a|)\cdot a$ is inside $B_r(x_0)$.
            However,
            $$\langle x_0+\frac{ra}{2|a|},a\rangle=d+\frac{r|a|}{2}\neq d.$$
            So $P$ has empty interior.
    \end{description}
    Then by definition, all hyperplanes are nowhere dense in $\mathbb{R}^n$.
    So $\mathbb{R}^n$ cannot be written as a countable union of hyperplanes.
\end{proof}

\begin{prop}[Well-approximable numbers]
    Let 
    $$\Lambda_n=\left\{x\in\mathbb{R}\,:\,\left|x-\frac{p}{q}\right|<\frac{1}{q^n}\text{
        for infinitely many }\frac{p}{q}\in\mathbb{Q}\right\}.$$
    $\Lambda_n$ is generic.
    Thus by Theorem~\ref{thm:5}, $\exists$ well-approximable irrationals, since
    $\mathbb{Q}$ is meager (Every countable set is meager).
\end{prop}
%Question: How dow we define well-approximable irrationals.
\begin{proof}
    By definition
    $$\Lambda_n^c=\left\{x\in\mathbb{R}\,:\,\left|x-\frac{p}{q}\right|\geq\frac{1}{q^n}\text{
        for all but finitely many }\frac{p}{q}\in\mathbb{Q}\right\}.$$
    If we can show that $\Lambda_n^c$ is meager, then since $R$ is a complete
    metric space, $\Lambda_n$ is generic.
    Now define
    $$
    F_q\coloneqq\left\{x\in\mathbb{R}\,:\,\forall p\in\mathbb{Z},
    \left|x-\frac{p}{q}\right|\geq\frac{1}{q^n}\right\},\quad
    E_q\coloneqq\bigcap_{q'\geq q}F_{q'}$$
    Since $F_{q'}$ is closed for all $q'$, $E_q$ is also closed for all $q$.
    Then 
    $$\Lambda_n^c=\bigcup_{q\in\mathbb{N}}E_q
    =\bigcup_{q\in\mathbb{N}}\bigcap_{q'\geq q}\left\{x\in\mathbb{R}\,:\,
    \forall
    p\in\mathbb{Z},\,\left|x-\frac{p}{q'}\right|\geq\frac{1}{(q')^n}\right\}.
    $$
    Now we need to show that $E_q$ is nowhere dense in $\mathbb{R}$.
    We know $\bar{E}_q=E_q$.
    But $E_q\cap\{p/q'\,:\,p,q'\in\mathbb{Z}, q'>q\}=\emptyset$.
    Since the latter set is a dense set in $\mathbb{R}$, then $E_q^c$ countains
    a dense set which implies the interier of $E_q$ is empty.
\end{proof}

    Furthermore, we can express $\Lambda_n$ in another way,
    $$\begin{aligned}
        \Lambda_n
        &=(\Lambda_n^c)^c\\
        &=\bigcap_{q\in\mathbb{N}}\bigcup_{q'\geq q}
        \left\{x\in\mathbb{R}\,:\,\exists
        p\in\mathbb{Z},\,\left|x-\frac{p}{q'}\right|<\frac{1}{(q')^n}\right\}\\
        &=\bigcap_{q\in\mathbb{N}}\bigcup_{q'\geq q}\bigcup_{p\in\mathbb{Z}}
        \left(\frac{p}{q'}-\frac{1}{(q')^n},\,\frac{p}{q'}+\frac{1}{(q')^n}\right)
    \end{aligned}$$
    The inner part is the union of intervals of width $2/(q')^n$ with spacing $1/q$.
    So heuristically, 
    $$\begin{aligned}
    &\Pr\left(x\in
    \bigcup_{p\in\mathbb{Z}}\left(\frac{p}{q'}-\frac{1}{(q')^n},\,\frac{p}{q'}+\frac{1}{(q')^n}\right)\right)
    &&\leq\frac{2/(q')^n}{1/q}=\frac{2}{(q')^{n-1}},\\
    %Question: Why is here less than or equal to?
    &\Pr\left(x\in
    \bigcup_{q'\geq q}
    \bigcup_{p\in\mathbb{Z}}\left(\frac{p}{q'}-\frac{1}{(q')^n},\,\frac{p}{q'}+\frac{1}{(q')^n}\right)\right)
    &&\leq\sum_{q'\geq
    q}^\infty\frac{2}{(q')^{n-1}}\\
    &\quad &&\leq\frac{2}{q^{n-\alpha}}\sum_{q'\geq
    q}^\infty\frac{(q')^{n-\alpha}}{(q')^{n-1}}\\
    &\quad &&\leq\frac{2}{q^{n-\alpha}}\sum_{q'\geq
    q}^\infty\frac{1}{(q')^{\alpha-1}}\\
    &\quad &&\leq\frac{C_n}{q^{n-\alpha}},
    \end{aligned}$$
    as long as $n> 2$ ($\alpha$ is also greater than $2$).
    So as $n\rightarrow+\infty$, the probability that $x\in\Lambda_n$ goes to zero.

\begin{defn}
    Let $X$ be a nonempty set and $(Y,d_y)$ be a metric space.
    Let $\{f_n\}$ be a sequence of functions from $X$ to $Y$, and let $f$ be a
    function from $X$ to $Y$.
    \begin{itemize}
        \item
            Say $f_n\rightarrow f$ pointwise if $\forall x\in X$ and $\forall
            \epsilon>0$, $\exists N=N(\epsilon,x)$, s.t.
            $\forall n\geq N$, $d_y(f_n(x),f(x))<\epsilon$.
        \item
            Say $\{f_n\}$ is pointwise Cauchy if $\forall x\in X$, $\forall
            \epsilon>0$, $\exists N=N(\epsilon,x)$, s.t. $\forall n,m>N$, 
            $d_y(f_n(x),f_m(x))<\epsilon$.
        \item
            Say $f_n\rightarrow f$ uniformly if $\forall \epsilon>0$, $\exists
            N=N(\epsilon)$, s.t. $\forall n>N$, $\forall x\in X$,
            $d_y(f_n(x)-f(x))<\epsilon$.
        \item
            Say $\{f_n\}$ is uniformly Cauchy if $\forall \epsilon>0$, $\exists
            N=N(\epsilon)$, s.t. $\forall n,m>N$, $\forall x\in X$, $d_y(f_n(x),
            f_m(x))<\epsilon$. That is to say
            $\lim_{n\rightarrow\infty}\sup_{x\in X}d(f_n(x),f(x))=0$.
    \end{itemize}
\end{defn}

\begin{note}
    Uniform convergence is much better than pointwise convergence.
\end{note}

\begin{defn}
    $f: X\rightarrow Y$ is a bounded function if $f(X)$ is a bounded subset of
    $Y$. (i.e.\ if $f(x)$ is contained in some metric ball $B_{r_f}(y_f)$ in Y)
\end{defn}

\begin{eg}
    The pointwise limit of a sequence of bounded functions need not be bounded.
    For example,
    $$f_n(x)=\left\{
        \begin{aligned}
            &x&&\text{for }|x|\leq N\\
            &N &&\text{for }x\geq N\\
            &-N\qquad&&\text{for }x\leq-N.
        \end{aligned}
    \right.$$
    Each $f_n$ is bounded in $[-N,N]$. However, its pointwise limit is $f(x)=x$,
    which is unbounded.
\end{eg}

\begin{defn}
    Let $X,Y$ be two nonempty metric spaces. Let 
    $$\mathcal{B}(X,Y)\coloneqq\{f:X\rightarrow Y\,:\,f(X)\text{ is a bounded
    set}\}.$$
\end{defn}

\begin{prop}
    The uniform limit of a sequence of bounded function is bounded.
\end{prop}
\begin{proof}
    Let $\{f_n\}$ be asequence in $\mathcal{B}(X,Y)$ and assume that
    $f_n\rightarrow f$ uniformly.
    By uniform convergence, $\exists n\in\mathbb{N}$, s.t. $\forall x\in X$,
    $d_Y(f(x),f_N(x))<1$.
    Since $f_N$ is a bounded function, $\exists y_0, k$, s.t. $f_N(x)\in
    B_k(y_0)$ for every $x$ in $X$.
    So $f(x)\in B_{k+1}(y_0)$ for every $x$ in $X$.
    Thus $f$ is bounded.
\end{proof}


\begin{prop}
    $\mathcal{B}(X,Y)$ is a metric sapce with metric
    $d_\mathcal{B}(f,g)=\sup_{x\in X}d(f(x),g(x))$.
    \begin{enumerate}
        \item If $f, g$ are bounded functions, then $d_{\mathcal{B}}(f,g)$ is
        finite.
        \item $d_\mathcal{B}$ is a metric on $\mathcal{B}(X,Y)$.
        \item Uniform convergent of a sequence in $\mathcal{B}(X,Y)$ is
        equivalent to metric convergence with respect to $d_\mathcal{B}$.
        \item If $Y$ is complete, then so is $\mathcal{B}(X,Y)$.
    \end{enumerate}
\end{prop}

\begin{defn}
    Let $X,Y$ be two nonempty metric spaces. Let
    $$\mathcal{C}(X,Y)\coloneqq\{\text{continuous functions from }X\text{ to }Y\}.$$
\end{defn}
\begin{eg}
    The pointwise limit of a sequence of continuous functions need not to be
    continuous.
    Let $f_n:[0,1]\rightarrow\mathbb{R}$, $f_n(x)=x^n$.
    But
    $$\lim_{n\rightarrow\infty}f_n(x)=\left\{
        \begin{aligned}
        &0,\quad&&\text{if }x\in[0,1)\\
        &1,\quad&&\text{if }x=1\\
        \end{aligned}
    \right.$$
    is not continuous.
\end{eg}

\begin{prop}[Honors HW]
    If $\{f_n\}$ is a sequence of functions on $\mathbb{R}$ and $f_n\rightarrow
    f$ pointwise, then the set of continuity points for $f$ is generic.
\end{prop}

\begin{thm}
    If $\{f_n\}$ is a sequence in $\mathcal{C}(X,Y)$ and $f_n\rightarrow f$
    uniformly. Then $f\in\mathcal{C}(X,Y)$.
\end{thm}
\begin{proof}
    Assume $f_n\rightarrow f$ uniformly.
    Let $x_0\rightarrow X$ and $\epsilon>0$.
    By uniform convergence, $\exists N\in\mathbb{N}$, s.t.
    $\forall x$, $d_Y(f(x),f_N(x))<\epsilon$.
    Since $f_N$ is continuous, $\exists\delta>0$, s.t. $\forall x$ with
    $d(x,x_0)<\delta$, we have $d_Y(f_N(x),f_N(x_0))<\epsilon$.
    Finally, by triangle inequality, $\forall x$ with $d_x(x,x_0)<\delta$,
    $$d_Y(f(x),f(x_0))\leq
    d_Y(f(x),f_N(x))+d_Y(f_N(x),f_N(x_0))+d_Y(f_N(x_0),f(x_0))<3\epsilon.$$
\end{proof}

\begin{defn}
    Let $X,Y$ be two nonempty metric spaces.
    Define
    $$\mathcal{C}^0(X,Y)\coloneqq\mathcal{C}(X,Y)\cap\mathcal{B}(X,Y).$$
    Then $\mathcal{C}^0(X,Y)$ is a metric space with metric
    $d_{\mathcal{C}^0}(f,g)\coloneqq d_{\mathcal{B}}(f,g)$.
\end{defn}
\begin{defn}
    For $Y=\mathbb{R}$, $X$ being any metric space, let
    $\mathcal{C}^0(X)\coloneqq\mathcal{C}^0(X,\mathbb{R})$ and define the norm
    (Question: why it is a norm?)
    $$\|f\|_{\mathcal{C}^0(X)}=\sup_{x\in X}|f(x)|.$$
\end{defn}

\begin{prop}
    Let $X,Y$ be two metric spaces. 
    $(\mathcal{C}^0(X,Y),d_{\mathcal{C}^0})$ is a metric space which is complete
    if $Y$ is complete.
\end{prop}
\begin{proof}
    Let $\{f_n\}$ be a Cauchy sequence in $\mathcal{C}^0(X,Y)$, then $\{f_n\}$
    is uniformly Cauchy in $\mathcal{C}^0(X,Y)$.
    Thus $\forall \epsilon>0$, $\exists N\in\mathbb{N}$, s.t. $\forall n,m\geq
    N$, $d_{\mathcal{C}^0}(f_n,f_m)<\epsilon$.
    Notice that 
    $d_{\mathcal{C}^0}(f_n,f_m)=\sup_{x\in X}d_Y(f_n(x),f_m(x))$.
    So in paticular, for any $x\in X$, $d_Y(f_n(x),f_m(x))<\epsilon$.
    Now fix $x\in X$.
    Then $\{f_n(x)\}$ is a Cauchy in $Y$.
    As $Y$ is complete, $\forall x\in X$, $\exists
    f(x)\coloneqq\lim_{n\rightarrow\infty}f_n(x)$.
    \begin{description}
        \item[Claim:]
            $f_n\rightarrow f$ uniformly.\\
            Let $\epsilon>0$.
            By uniform Cauchyness, we may choose $N\in\mathbb{N}$, s.t.
            $\forall n,m\geq N$, $\forall x\in X$,
            $d_Y(f_n(x),f_m(x))<\epsilon$.
            Now fix $n\geq N$, $x\in X$.
            Choose $m_x$ s.t. $m_x\geq N$ and $d_Y(f_{m_x}(x),f(x))<\epsilon$.
            Then
            $$d_Y(f_n(x),f(x))\leq
            d_Y(f_n(x),f_{m_x}(x))+d_Y(f_{m_x}(x),f(x))<2\epsilon.$$
            Note, it is okay that $m_x$ depends on $x$, since it doesn't appear
            on either side of the inequality.
            Since $d_Y(f_n(x),f(x))<2\epsilon$ for all $x\in X$,
            $$d_{\mathcal{C}^0}(f_n(x),f)\leq 2\epsilon<3\epsilon.$$
            Thus, $f_n\rightarrow f$ in $\mathcal{C}^0$.
    \end{description}
\end{proof}

\begin{thm}
    There exists a nowhere differentiable (not differentiable at any point)
    continuous function $f\in\mathcal{C}^0([0,1])$.
\end{thm}
\begin{proof}
    Since $\mathbb{R}$ is complete, $\mathcal{C}^0([0,1])$ is a complete metric
    space and thus it is not meager.
    It sufficies to prove that
    $$F\coloneqq\{f\in\mathcal{C}^0([0,1])\,:\,\exists x_0,\text{ s.t.
    }f'(x_0)\text{ exists}\}$$
    is meager, i.e.\ a countable union with nowhere dense sets.
    It then suffices to prove that $F$ is contained in a countable union of
    nowhere dense sets.
    \begin{description}
        \item[Claim 1:]
            $F\subseteq\cup_{n=1}^\infty F_n$, where
            $$F_n\coloneqq\{f\in\mathcal{C}^0([0,1])\,:\,\exists
            x_0\in[0,1],\text{ s.t. }|f(x)-f(x_0)|\leq n|x-x_0|,\forall
            x\in[0,1]\}.$$
            If $f\in F$, then $\exists x_0$, s.t. 
            $$f'(x_0)=\lim_{h\rightarrow 0}\frac{f(x_0+h)-f(x_0)}{h}$$
            exists.
            Thus, there exists $\delta>0$, s.t. $\forall x\in[0,1]$ with
            $|x-x_0|<\delta$,
            $$|f(x)-f(x_0)|\leq(|f'(x_0)|+1)|x-x_0|.$$
            For $x\in[0,1]$ with $|x-x_1|\geq\delta$, we have
            $$|f(x)-f(x_0)|\leq |f(x)|+|f(x_0)|\leq
            2\|f\|_{\mathcal{C}^0}\cdot\frac{|x-x_0|}{|x-x_0|}
            \leq\frac{2\|f\|_{\mathcal{C}^0}}{\delta}\cdot|x-x_0|.$$
            Finally, for any $n\geq |f'(x_0)|+1+2\|f\|_{\mathcal{C}^0}/\delta$,
            we have $f\in F_n$.
        \item[Claim 2:]
            $F_N$ is closed.\\
        \item[Claim 3:]
            $F_N$ is nowhere dense.\\
            Since $F_N$ is closed, it suffices to prove that $F_N$ has an empty
            interior, i.e.\ that $\forall f\in F_N$, $\forall \epsilon>0$,
            $\exists g\in \mathcal{C}^0([0,1])$, s.t.
            $\|f-g\|_{\mathcal{C}^0([0,1])}<\epsilon$ and $g\notin F_N$.
            Let $f\in F_N$ and $\epsilon>0$.
            The idea is to find $g$, piecewise linear, such that the slopes of
            linear parts of $g$ has absolute value $>N$.

            Since $f$ is continuous and $[0,1]$ is compact, $f$ is uniformly
            continuous. So there exists $\delta>0$, s.t. $|x-y|\leq\delta$,
            $|f(x)-f(y)|\leq\epsilon$.
            Choose $n\in\mathbb{N}$, s.t. $1/n<\delta$ and
            $2\epsilon/(1/n)>1000N$.
            Set $x_j=j/n$, $0\leq j\leq n$.
            Define $$g(x_j)\coloneqq f(x_j)+(-1)^j\epsilon,$$
            and make $g$ linear in $x\in(x_j,x_{j+1})$, for $j=0,\ldots,n$.
            Note, if $x\in[x_j,x_{j+1}]$, then $x=(1-\theta)x_j+\theta x_{j+1}$
            for some $0\leq\theta\leq1$, and this implies
                $g(x)=(1-\theta)g(x_j)+\theta g(x_{j+1})$.
            \begin{description}
                \item[Subclaim 1:]
                $\|g-f\|_{\mathcal{C}^0([0,1])}<3\epsilon$.\\
                Suffices to show that $\forall j$ and $\forall x\in[x_j,x_{j+1}]$,
                $|g(x)-f(x)|<3\epsilon$.
                Write $x=(1-\theta)x_j+\theta x_{j+1}$, with $0\leq\theta\leq
                1$.
                Then
                $$\begin{aligned}
                |g(x)-f(x)|&=|(1-\theta)(g(x_j)-f(x_j))+(1-\theta)(f(x_j)-f(x))\\
                &\qquad\qquad\qquad+\theta(g(x_{j+1})-f(x_j))+\theta(f(x_j)-f(x))|\\
                &\leq
                (1-\theta)|g(x_j)-f(x_j)|+(1-\theta)|f(x_j)-f(x)|\\
                &\qquad\qquad\qquad+\theta|g(x_{j+1})-f(x_j)|+\theta|f(x_j)-f(x)|\\
                &\leq
                (1-\theta)\epsilon+(1-\theta)\epsilon+\theta\epsilon+\theta\epsilon\\
                &=2\epsilon<3\epsilon
                \end{aligned}$$
                \item[Claim 2:]
                The slopes of $g$ have absolute value greater than $N$:\\
                Suffices to prove $$\frac{|g(x_{j+1})-g(x_j)|}{1/n}>N.$$
                Indeed,
                $$|g(x_{j+1})-g(x_j)|\geq2\epsilon-|f(x_{j+1})-f(x_j)|\geq
                \epsilon,$$
                because $|x_{j+1}-x_j|\leq\delta$.
                Since $n\epsilon>500N>N$, done.
            \end{description}
            Finally, observe that $g\notin F_N$ since any $x_0$ belongs to some
            $[x_j,x_{j+1}]$ and $|g(x)-g(x_0)|>N|x-x_0|$ for $x_0\neq
            x\in[x_j,x_{j+1}]$.
    \end{description}
\end{proof}

%Uniform convergence and integration
\section*{Uniform Convergence and Integration}
\begin{defn}
    Let $f:[a,b]\rightarrow\mathbb{R}$ be a bounded function.
    \begin{itemize}
        \item
            A partition of $[a,b]$ is a finite set
            $$P=\{a=x_0<x_1<\cdots<x_N=b\}.$$
        \item
            Define intervals $I_j\coloneqq[x_{j-1},x_j]$ for $j=1,\ldots,N$, with
            lengths $\Delta x_j\coloneqq x_j-x_{j-1}$.
        \item
            Upper Riemann sums:
            $$U(f,p)\coloneqq\sum_{j=1}^N M_j(f,P)\Delta x_j,$$
            where $M_j(f,p)=\sup_{x\in I_j}f(x)$.
        \item
            Lower Riemann sums:
            $$L(f,p)\coloneqq\sum_{j=1}^N m_j(f,P)\Delta x_j,$$
            where $m_j(f,p)=\inf_{x\in I_j}f(x)$.
    \end{itemize}
\end{defn}

\begin{thm}
    $f$ is Riemann integrable if and only if $\forall \epsilon>0$, there exists an
    partition $P$, s.t. $U(f,P)-L(f,P)<\epsilon$.
    In this case,
    $$\int_a^b f(x)\,\mathrm{d}x=\inf_P U(f,P)=\sup_P L(f,P).$$
\end{thm}

\begin{eg}
    Pointwise limit of Riemann integrable functions need not be Riemann
    integrable.
    Let $f_n(x)$ be a function from $[0,1]$ to $\mathbb{R}$ defined as following
    $$f_n(x)=\left\{
        \begin{aligned}
        &1,\quad&&\text{if }x\in\mathbb{Q}\cap[0,1]\ \text{with denominator of x
        at most }n\\
        &0,&&\text{otherwise}
        \end{aligned}
    \right.$$
    $f_n(x)$ is Riemann integrable since it is piecewise linear.
    Consider the limit
    $$\lim_{n\rightarrow\infty}f_n(x)=f(x)=\left\{
        \begin{aligned}
        &1,\quad&&\text{if }x\in\mathbb{Q}\cap[0,1]\\
        &0,&&\text{if }x\in[0,1]\setminus\mathbb{Q}
        \end{aligned}
    \right.$$
    For any partition $P$, $U(f,P)=1$ and $L(f,P)=0$. We see the limit $f(x)$ is
    not Riemann integrable.
\end{eg}

\begin{thm}
    Let $\{f_n\}$ be a sequence of Riemann integrable functions on $[a,b]$ and
    assume $f_n\rightarrow f$ uniformly.
    Then $f$ is Riemann integrable and
    $$\int_a^b
    f(x)\,\mathrm{d}x=\lim_{n\rightarrow\infty}\int_a^bf_n(x)\,\mathrm{d}x.$$
\end{thm}

\begin{note}
    Under the assumption of pointwise convergence, the above formula can fail
    even if the limit $f$ is Riemann integrable.\\
    Under the assumption of uniform convergence, this is a total mess if $[a,b]$
    is replaced by $[a,\infty)$ and $\int_a^b$ is replaced by $\int_a^\infty$.\\
\end{note}

%exam: stops at Friday's lecture,
%format: 3-4 problems, 1 problem is from homework, 1 from class, 1 new.

\section*{Uniform Convergence and Differentiation}
\begin{eg}
    Consider the sequence 
    $$f_n(x)=\sqrt{\frac{1}{n}+x^2},\quad x\in\mathbb{R}.$$
    \begin{description}
        \item[Claim:] $f_n\rightarrow f$ uniformly, where $f=|x|$, on
        $\mathbb{R}$.\\
        Let $\epsilon>0$. Let $N=\lceil1/\epsilon\rceil$. 
        Then $\forall n>N$ and $\forall x$,
        we have
        $$\left|f_n(x)-|x|\right|<\frac{1}{n}<\epsilon.$$
    \end{description}
    Furthermore,  each $f_n$ is differentiable and
    $$f_n'(x)=\frac{x}{\sqrt{\frac{1}{n}+x^2}}.$$
    However, $f_n'\rightarrow g$ pointwise, where
    $$g(x)=\left\{
        \begin{aligned}
            &-1,\qquad&&x<0,\\
            &1,\qquad&&x>0,\\
            &0,\qquad&&x=0.\\
        \end{aligned}
    \right.$$
    Also we observe that the limiting function $f$ is not differentiable.
\end{eg}
\begin{defn}
    Let $I$ be an interval with nonempty interior.
    Let 
    $$\mathcal{C}^k(I)\coloneqq\{f:I\rightarrow\mathbb{R}\,:\,f\text{ is
    }k\text{-`times' differentiable and }f^{(j)}\in \mathcal{C}^0(I),\,0\leq j\leq
    k\}.$$
    Define the norm
    $$\|f\|_{\mathcal{C}^k(I)}\coloneqq\sum_{j=0}^k\|f^{(j)}\|_{\mathcal{C}^0(I)}.$$
\end{defn}
\begin{prop}[HW]
    $\exists A=A(I)$, s.t. $\|f\|_{C^k(I)}\leq
    A(\|f\|_{C^0(I)}+\|f^{(k)}\|_{C^0(I)})$.
    Further, this $A$ can be independent of $I$.
\end{prop}

\begin{thm}
    $\mathcal{C}^k(I)$ is a complete metric space.
\end{thm}
\begin{proof}
    Prove by induction.
    First, we know that $\mathcal{C}^0(I)$ is complete.
    We want to deduce completeness of $\mathcal{C}^{k+1}(I)$ from completeness of
    $\mathcal{C}^k(I)$.\\
    Now we assume $\mathcal{C}^k(I)$ is complete and let $\{f_n\}$ be a Cauchy
    sequence in $\mathcal{C}^{k+1}(I)$.
    Notice $\forall n,m$,
    $$\|f_n-f_m\|_{\mathcal{C}^k(I)}+\|f_n^{(k+1)}-f_m^{(k+1)}\|_{\mathcal{C}^0(I)}=\|f_n-f_m\|_{\mathcal{C}^{k+1}(I)},$$
    so $\{f_n\}$ is a Cauchy sequence in $C^k(I)$.  
    Then by hypothesis, $\exists f\in \mathcal{C}^k(I)$, s.t. 
    $\|f_n-f\|_{\mathcal{C}^k(I)}\rightarrow 0$ as $n\rightarrow\infty$.
    Since $\{f_n^{(k+1)}\}$ is Cauchy in $\mathcal{C}^0(I)$, $\exists g\in \mathcal{C}^0(I)$, s.t.
    $f_n^{(k+1)}\rightarrow g$ uniformly on $I$.
    If we can show that $f\in \mathcal{C}^{k+1}(I)$ and $f^{(k+1)}=g$, then
    $$\begin{aligned}
    \|f_n-f\|_{\mathcal{C}^{k+1}(I)}
    &=\|f_n-f\|_{\mathcal{C}^k(I)}+\|f_n^{(k+1)}-f^{(k+1)}\|_{\mathcal{C}^0(I)}\\
    &=\|f_n-f\|_{\mathcal{C}^k(I)}+\|f_n^{(k+1)}-g\|_{\mathcal{C}^0(I)}
    \end{aligned}$$
    goes to 0 as $n\rightarrow\infty$.
    For this, it suffices to prove that $f^{(k)}$ is differentiable and
    $(f^{(k)})'=g$.
    Fix $x_0\in I$.
    Then $\forall x\in I$, by the fundamental theorem of calculus, we have
    $$f_n^{(k)}(x)=f_n^{(k)}(x_0)+\int_{x_0}^x(f_n^{(k)})'(y)\,\mathrm{d}y.$$
    Since $f_n\rightarrow f$ in $\mathcal{C}^k$,
    $\lim_{n\rightarrow\infty}f_n^{(k)}(x)=f^{(k)}(x)$ for all $x\in I$.
    Since $f_n^{(k+1)}=(f_n^{(k)})'\rightarrow g$ uniformly, we know
    $$\lim_{n\rightarrow\infty}\int_{x_0}^x(f_n^{(k)})'(y)\,\mathrm{d}y=\int_{x_0}^x\lim_{n\rightarrow\infty}(f_n^{(k)})'(y)\,\mathrm{d}y=\int_{x_0}^xg(y)\,\mathrm{d}y.$$
    Finally, by linearity of limits,
    $$f^{(k)}(x)=f^{(k)}(x_0)+\int_{x_0}^xg(y)\,\mathrm{d}y.$$
    Since $g$ is continuous, the fundamental theorem of calculus says
    $\int_{x_0}^xg(y)\,\mathrm{d}y$ is differentiable with derivative $g(x)$.
    Then we can conclude that $f^{(k)}$ is differentiable and $f^{(k)}=g$.
\end{proof}

\begin{prop}[HW]
    Let $\{f_n\}$ be a sequence in $\mathcal{C}^k(I)$.
    Assume $\{f_n^{(k)}\}$ is Cauchy in $\mathcal{C}^0(I)$ and $\exists x_0\in
    I$, s.t. $\forall j= 0,\ldots,k-1$, $\{f_n^{(j)}(x_0)\}$ is a Cauchy sequence.
    Then $\{f_n\}$ is convergent in $\mathcal{C}^k(I)$.
\end{prop}

\begin{thm}[7.17]
    Suppose $\{f_n\}$ is a sequence of functions, differentiable on $[a,b]$ and such
    that $\{f_n(x_0)\}$ converges for some point $x_0$ on $[a,b]$.
    If $\{f_n'\}$ converges uniforly on $[a,b]$, then $\{f_n\}$ converges uniformly
    on $[a,b]$, to a function $f$, and
    $$f'(x)=\lim_{n\rightarrow\infty}f_n'(x)\quad(a\leq x\leq b).$$
\end{thm}

\begin{eg}
    Let 
    $$f(x)=\left\{
       \begin{aligned}
           &x^2\sin 1/x,\qquad &&x\neq0\\
           &0,&&x=0.
       \end{aligned}
    \right.$$
    It has derivative
    $$f'(x)=\left\{
        \begin{aligned}
            &2x\sin 1/x-\cos 1/x,\qquad &&x\neq0\\
            &0,&&x=0
        \end{aligned}
    \right.$$
    and $f'(x)$ is discontinuous.
    So there exists some functions which has derivative at every point, but the
    derivative is not continuous.
\end{eg}

\begin{recall}
    Let $(X,d)$ be a complete metric space. 
    A subset $K\subseteq X$ is compact if and only if $K$ is closed and totally
    bounded.
\end{recall}
\begin{corollary}
    $\mathcal{F}\subseteq\mathcal{C}^0(X)$ is compact if and only if
    $\mathcal{F}$ is closed and totally bounded.
\end{corollary}
But what does totally bounded mean for $\mathcal{C}^0(X)$?
We want a simpler characterization of totally boundedness.

\begin{defn}
    Let $\mathcal{F}\subseteq \mathcal{C}(X)$.
    \begin{itemize}
        \item $\mathcal{F}$ is pointwise bounded if $\forall x\in X$, $\exists
            M_x$, s.t. $\forall f\in\mathcal{F}$, $|f(x)|\leq M_x$ (i.e. $\forall
            x\in X$, $\{f(x): f\in\mathcal{F}\}$ is a bounded set).
        \item $\mathcal{F}$ is equicontinuous if $\forall \epsilon>0$, $\forall
        x\in X$, $\exists \delta=\delta(x,\epsilon)>0$, s.t.  $\forall
        f\in\mathcal{F}$ and $\forall y\in\mathcal{B}_\delta(x)$,
        $|f(x)-f(y)|<\epsilon$.
        \item $\mathcal{F}$ is uniformly equicontinuous if $\forall
            \epsilon>0$, $\exists \delta=\delta(\epsilon)>0$, s.t. $\forall
            f\in\mathcal{F}$ and $\forall x,y\in X$ with $d(x,y)<\delta$,
            $|f(x)-f(y)|<\epsilon$.
    \end{itemize}
\end{defn}
\begin{eg}
    $\mathcal{F}\coloneqq\{f\in\mathcal{C}^0([0,1]):f\text{ if differentiable
    on }(0,1)\text{ and }|f'(x)|\leq 1, \forall x\in(0,1)\}$ is uniformly
    equicontinuous.
\end{eg}
\begin{proof}
    $\forall f\in\mathcal{F}$, $x,y\in[0,1]$, $|f(x)-f(y)|\leq |x-y|$ by mean
    value theorem.
    So $\forall \epsilon>0$, pick $\delta=\epsilon$.
    Then $|f(x)-f(y)|<\epsilon$ for all $f\in\mathcal{F}$ and for all $x,y\in[0,1]$ with $|x-y|<\delta$.
\end{proof}

\begin{eg}
    $f_n(x)=x^n$ defined on $[0,1]$.
    Then $\mathcal{F}\coloneqq\{f_n:n\in\mathbb{N}\}$ is a pointwise
    bounded, but not a equicontinuous subset of $\mathcal{C}^0([0,1])$.
\end{eg}
\begin{proof}
    For any $x\in [0,1]$, $\{f(x):f\in\mathcal{F}\}\subseteq[0,1]$.
    So $\mathcal{F}$ is pointwise bounded.
    Fix $x=1$ and $\epsilon=0.5$.
    For any $\delta>0$, there exists $N$ sufficiently large such that for all
    $n>N$, $f_n(x-\delta/2)<0.5$.
    Thus, $f_n(x)$ is not equilcontinous.
\end{proof}

\begin{prop}
    If $K$ is compact, then $\mathcal{C}^0(K)=\mathcal{C}(K)$.
    ($\mathcal{C}^0$ means bounded continuous functions, while $\mathcal{C}$
    means continuous functions)
\end{prop}
\begin{eg}
    Show by example that a pointwise bounded subset of $\mathcal{C}^0(K)$ need
    not be uniformly pointwise bounded (i.e.\ a bounded subset of
    $\mathcal{C}^0(K)$).
    $$f_n(x)=\left\{
        \begin{aligned}
        &n^2x,&&0\leq x\leq 1/n\\
        &2n-n^2x,\qquad&&1/n<x\leq2/n\\
        &0&&2/n<x\leq1\\
        \end{aligned}
    \right.$$
\end{eg}

\begin{prop}
    If $K$ is compact, then $\mathcal{F}\subseteq\mathcal{C}^0(K)$ is
    equicontinuous if and only if $\mathcal{F}$ is uniformly equicontinuous.
\end{prop}
\begin{proof}
    $\Leftarrow:$ is immediate.\\
    $\Rightarrow:$ Assume $K$ is compact and
    $\mathcal{F}\subseteq\mathcal{C}^0(K)$ is equicontinuous.
    Let $\epsilon>0$.
    Then $\forall x\in K$, $\exists \delta_x>0$, s.t.
    $\forall y\in B_{\delta_x}(x)$, $\forall f\in\mathcal{F}$, $|f(x)-f(y)|<\epsilon/2$.
    Since $K\subseteq \cup_{x\in K}B_{\delta_x}(x)$, $\exists x_1,\ldots,x_N$,
    s.t.
    $K\subseteq\cup_{j=1}^N B_{\delta_{x_j}}(x_j)$.
    \begin{description}
        \item[Claim: ]
        $\exists \delta>0$, s.t. $\forall y,z\in K$, if $d(y,z)<\delta$, then
        $y,z\in B_{\delta_{x_j}}(x_j)$.\\
        Suppose not. Then there exists sequences $\{y_n\}$ and $\{z_n\}$ such
        that $d(y_n,z_n)\rightarrow0$, but $y_n$ and $z_n$ never belong to the
        same $B_{\delta_{x_j}}(x_j)$.
        Each $z_n$ lives in some $\mathcal{B}_{\delta_{x_j}}(x_j)$ and since
        there are only
        finitely many such balls, there must be a ball that contains infinitely
        many $z_n$.
        Passing to a subsequence, we may assume $\exists j_0$, s.t. $z_n\in
        B_{\delta_{x_{j_1}}}(x_{j_1})$ for all $n$.
        Similarly, passing to a further subsequence, we may assume $y_n\in
        B_{\delta_{x_{j_2}}}(x_{j_2})$ for all $n$.
        Since $\{z_n\}$ is in $K$, which is compact, passing to a further
        subsequence, we may assume $z_n\rightarrow z$.
        Since $d(z_n,y_n)\rightarrow0$, $y_n\rightarrow z$.
        Notice $\exists j$, s.t. $z\in B_{\delta_{x_j}}(x_j)$ ($z$ is not
        necessarily in $B_{\delta_{x_{j_1}}}(x_{j_1})$).
        For $n$ sufficiently large, $y_n$ and $z_n$ are both in
        $B_{\delta_{x_j}}(x_j)$.
        Contradiction.
    \end{description}
    Then we have
    $$|f(y)-f(z)|\leq|f(y)-f(x_j)|+|f(x_j)-f(z)|<\epsilon$$
    for any $f\in\mathcal{F}$.
\end{proof}

\begin{thm}[Arzela-Ascali Theorem]
    If $K$ is a compact metric space, then $\mathcal{F}\subseteq
    \mathcal{C}^0(K)$ is totally bounded if and only if $\mathcal{F}$ is
    pointwise bounded and equicontinuous.
\end{thm}
\begin{proof}
    \begin{description}
    \item[$\Rightarrow:$]
        Assume $\mathcal{F}\subseteq\mathcal{C}^0(K)$ is totally bounded.\\
        \textbf{Pointwise bounded:}
        Since $\mathcal{F}$ is totally bounded, $\exists N$ and
        $f_1,\ldots,f_n\in\mathcal{F}$, s.t. $\mathcal{F}\subseteq\cup_{j=1}^N
        B_1(f_1)$.
        Then $\forall x$ and $\forall f\in\mathcal{F}$,
        $$|f(x)|<\sum_{j=1}^N|f_j(x)|+1\leq\sum_{j=1}^N\|f_j\|_{\mathcal{C}^0(K)}+1.$$
        \textbf{Equicontinuous:}
        Let $\epsilon>0$.
        Since $\mathcal{F}$ is totally bounded, $\exists N$ and
        $f_1,\ldots,f_n\in\mathcal{F}$, s.t. $\mathcal{F}\subseteq\cup_{j=1}^N
        B_{\epsilon/3}(f_1)$.
        Since each $f_j$ is uniform continuous (being continuous on a compact
        set), $\exists \delta_j>0$, s.t. $\forall x,y\in K$ with
        $d(x,y)<\delta_j$, we have $|f_j(x)-f_j(y)|<\epsilon/3$.
        Let $\delta\coloneqq\min\{\delta_1,\ldots,\delta_N\}$.
        Let $f\in\mathcal{F}$.
        Then $\exists j$, s.t. $\|f-f_j\|_{\mathcal{C}^0}<\epsilon/3$.
        Finally, if $d(x,y)<\delta$,
        $$|f(x)-f(y)|\leq|f(x)-f_j(x)|+|f_j(x)-f_j(y)|+|f_i(y)-f(y)|<\epsilon.$$
        Thus $\mathcal{F}$ is uniform equicontinuous.
    \item[$\Leftarrow:$]
        Assume $\mathcal{F}$ is pointwise bounded and equicontinouus.
        Let $\epsilon>0$.
        By proposition, $\mathcal{F}$ is uniformly equicontinous, so
        $\exists\delta>0$, s.t. $\forall f\in \mathcal{F}$, $\forall x,y\in K$
        with $d(x,y)<\delta$, $|f(x)-f(y)|<\epsilon/4$.
        Since $K$ is compact, $K$ is totally bounded.
        So $\exists N$ and $x_1,\ldots,x_N\in K$, s.t. $K\subseteq\cup_{j=1}^N
        B_{\delta}(x_j)$.
        The idea is to discretize the functions in $\mathcal{F}$.
        Consider
        $P\coloneqq\{(f(x_1),f(x_2),\ldots,f(x_N)):f\in\mathcal{F}\}\subseteq\mathbb{R}^N$.
        By pointwise boundedness, for each $j$, $\exists M_j$, s.t.
        $\{f(x_j):f\in\mathcal{F}\}\subseteq[-M_j,M_j]$.
        Therefore
        $$P\subseteq[-M_1,M_1]\times[-M_2,M_2]\times\cdots\times[-M_N,M_N],$$
        which is a bounded (hence totally bounded) subset of $\mathbb{R}^N$.
        So $P$ is totally bounded.
        Then $\exists L$ and $y_1,\ldots,y_L\in P$, s.t. $P\subseteq\cup_{j=1}^L
        B_{\epsilon/4}(y_j)$. (Note: $B_{\epsilon/4}(y_j)$ is the ball in
        Euclidean space.)
        For each $y_j$, by definition, $\exists f_j\in\mathbb{F}$, s.t.
        $(y_j)_i=f_j(x_i)$, $i=1,\ldots,N$.
        Thus, $\forall f\in\mathcal{F}$, $\exists j$, s.t. $1\leq j\leq L$ and
        $$|f(x_i)-f_j(x_i)|\leq|(f(x_1),\ldots,f(x_N))-(f_j(x_1),\ldots,f_j(x_N))|<\epsilon/4,\quad
        i=1,\ldots,N$$
        For any $x\in K$, by totally boundedness of $K$, $\exists i\in1,\ldots,N$,
        s.t. $x\in
        B_\delta(x_i)$.
        So
        $$|f(x)-f_j(x)|\leq
        |f(x)-f(x_i)|+|f(x_i)-f_j(x_i)|+|f_j(x_i)-f_j(x)|<3\epsilon/4.$$
        Taking the supremum over $x$, $\|f_j-f\|_{\mathcal{C}^0(K)}\leq
        3\epsilon/4<\epsilon$.
        Thus, $f\in B_\epsilon(f_j)$ and we have
        $\mathcal{F}\subseteq\cup_{j=1}^L B_\epsilon(f_j)$.
    \end{description}
\end{proof}
\begin{corollary}
    Let $K$ be a compact metric space.
    \begin{enumerate}
        \item $\mathcal{F}\subseteq \mathcal{C}^0(K)$ is compact if and only if
        $\mathcal{F}$ is closed, pointwise bounded and equicontinuous.
        \item $\mathcal{F}\subseteq\mathcal{C}^0(K)$ is precompact if and only
        if $\mathcal{F}$ is pointwise bounded and equicontinuous.
        \item Let $\{f_n\}$ be a pointwise bounded equicontinuous sequence in
        $\mathcal{C}^0(K)$.
        Then $\{f_n\}$ has a uniformly convergent subsequence.
    \end{enumerate}
\end{corollary}

\begin{prop}[]
    For $k\in\mathbb{N}$ and $I$ compact, a subset of $\mathcal{C}^k(I)$ is
    totally bounded if and only each of its $i$th derivatives, $i=0,1,\ldots,k$, is
    pointwise bounded and equicontinuous.
\end{prop}

% Part Two
% Second Midterm Starts from here
\begin{thm}[(Baby) Stone-Weierstress Theorem]
    Let $f:[a,b]\rightarrow\mathbb{R}$ be a continuous function.
    There exists a sequence $\{P_n\}$ of polynomials, s.t. $P_n\rightarrow f$
    uniformly on $[a,b]$.
\end{thm}
    Note that even if $f$ has derivatives of all orders, it probably doesn't work if we
    just take $P_n$ equal to the $n$-th Taylor polynomial.
    Also, unless $f$ itself is a polynomial, the degress of $P_n$ must tend to
    infinity.
\begin{proof}
    First, we do some reductions:
    \begin{enumerate}
        \item We assume $[a,b]=[0,1]$.
        If this is not true, we can approximate $g(x)\coloneqq
        f(\frac{x-a}{b-a})$ and replace $P_n(x)$ by $P_n((b-a)x+a)$.
        \item We assume $f(0)=f(1)=0$.
        If this is not the case, we approximate $g(x)=f(x)-f(0)-x(f(1)-f(0))$
        and replace $P_n(x)$ by $P_n(x)+f(0)+x(f(1)-f(0))$.
    \end{enumerate}
    Now, we extend $f$ to all of $\mathbb{R}$ by setting $f(x)=0$ off of $[0,1]$.
    (Warning: We are only approximating $f$ on $[0,1]$, even though the domain
    of $f$ is larger.)
    Then $f$ is continous and thus uniformly continuous on $\mathbb{R}$.
    Define $$Q_n(x)\coloneqq c_n(1-x^2)^n,$$ where $c_n$ is chosen s.t.
    $$\int_{-1}^1Q_n(x)\,\mathrm{d}x=1,$$
    i.e.\  $c_n=\left(\int_{-1}^1(1-x^2)^n\,\mathrm{d}x\right)^{-1}$.
    Let 
    $$P_n(x)\coloneqq\int_0^1f(t)Q_n(x-t)\,\mathrm{d}t=f*Q_n(x)
    =\int_{x-1}^xf(x-t)Q_n(t)\,\mathrm{d}t.$$
    We need to prove that $P_n$ is a polynomial and $P_n\rightarrow f$
    uniformly on $[0,1]$.
    \begin{description}
        \item[Claim 1:] $P_n$ is a polynomial of degree $\leq2n$.
        \begin{proof}[Subproof]
            Expand the expression for $Q_n(x-t)$, 
            $$\begin{aligned}
            Q_n(x-t)&=c_n(1-(x-t)^2)^n=\sum_{j=0}^nc_n\binom{n}{j}(-1)^j(x-t)^{2j}\\
            &=\sum_{j=0}^n\sum_{k=0}^{2j}c_n \binom{n}{j}\binom{2j}{k}(-1)^{j+k}x^{2j-k}t^k\\
            &=\sum_{m=0}^{2n}\left(\sum_{m\leq 2j\leq 2n}c_n\binom{n}{j}
            \binom{2j}{2j-m}(-1)^{j-m}x^mt^{2j-m}\right)
            \end{aligned}$$
            So $P_n(x)=\sum_{m=0}^{2n}a_{n,m}x^m$, where
            $$a_{n,m}=\sum_{m\leq 2j\leq
            2n}c_n\binom{n}{j}\binom{2j}{2j-m}(-1)^{j-m}\int_0^1f(t)t^{2j-m}\,\mathrm{d}t.$$
        \end{proof}
        \begin{proof}[Subproof (An alternative approach)]
            For $k\in\mathbb{N}$, $P_n$ is $k$ times differentiable and
            $$P_n^{(k)}(x)=\int_0^1f(t)Q_n^{(k)}(x-t)\,\mathrm{d}t.$$
            In particular, since $Q_n$ is a polynomial of degree $2n$,
            $$p_n^{(2n+1)}(x)=\int_0^1f(t)\cdot 0\,\mathrm{d}t=0.$$
            So $P_n$ is a polynomial of degree at most $2n$.
            By induction, it sufficies to prove the following proposition, which
            sometimes is of independent interest.
        \end{proof}
        \begin{prop}
            Let $g\in\mathcal{C}^1(\mathbb{R})$ and let
            $f\in\mathcal{C}^0(\mathbb{R})$ with $f=0$ out of $[0,1]$.
            Then the convolution
            $$h(x)\coloneqq\int_0^1 f(t)g(x-t)\,\mathrm{d}t$$
            is also in $\mathcal{C}^1(\mathbb{R})$, where its derivative is 
            $$h'(x)=\int_0^1f(t)g'(x-t)\,\mathrm{d}t.$$
            Note: Since $f(x)=0$ off of $[0,1]$, we can also replace the proper
            integral $\int_0^1$ by the inproper integral $\int_{-\infty}^\infty$.
        \end{prop}
        \begin{proof}[Proof of Proposition]
            First, observe that $h$ is bounded in $\mathcal{C}^1(\mathbb{R})$.
            This is because
            $$|h(x)|=\left|\int_0^1f(t)g(x-t)\,\mathrm{d}x\right|\leq
            \int_0^1\left| f(t)g(x-t)\right| \,\mathrm{d}t
            \leq\|f\|_{\mathcal{C}^0}\|g\|_{\mathcal{C}^1}.$$
            By a similar argument, we observe that
            $$|h'(x)|\leq\|f\|_{\mathcal{C}^0}\|g\|_{\mathcal{C}^1}.$$
            In addition, we claim that $h'(x)$ is continuous.
            Let $x_n\rightarrow x$.  Define
            $$\varphi_n(t)\coloneqq f(t)g'(x_n-t)\quad\text{and}\quad
            \varphi(t)\coloneqq f(t)g'(x-t).$$
            Without loss of generality, assume $x_n\in[x-1,x+1]$ for all $n$.
            Observe that every $\varphi_n$ and $\varphi$ lives on $[0,1]$.
            For any $t\in[0,1]$, $x_n-t\in[x-2,x+1]$.
            Since $g'$ is continuous, it is uniform continuous on $[x-2,x+1]$.
            Hence $\forall \epsilon>0$, $\exists N$, s.t. $\forall n\geq N$,
            $\forall t\in[0,1]$, $|g'(x_n-t)-g'(x-t)|<\epsilon$ (because
            $|x_n-x|<$ some $\delta$).
            Thus, $\varphi_n\rightarrow\varphi$ uniformly on $[0,1]$.
            Then we have
            $$\lim_{x_n\rightarrow x}h'(x_n)
            =\lim_{x_n\rightarrow x}\int_0^1\varphi_n(t)\,\mathrm{d}t=
            \int_0^1\varphi(t)\,\mathrm{d}t
            =h'(x).$$
        \end{proof}
        \item[Claim 2:] $P_n\rightarrow f$ uniformly on $[0,1]$.
            \begin{description}
                \item[Subclaim 1:] $c_n\leq\sqrt{n}$, $\forall n$.\\
                Let $g(y)=(1-y)^n$.
                Then $g''(y)=n(n-1)(1-y)^{n-2}\geq0$ for $y\in[0,1]$.
                By Taylor's theorem, for $y\in[0,1]$, 
                $$\begin{aligned}
                    g(y)&=g(0)+g'(0)y+\frac{1}{2}g''(ty)y^2,\quad \text{for some
                }t\in[0,y]\subseteq[0,1]\\
                    &\geq g(0)+g'(0)y\\
                    &=1-ny.
                \end{aligned}$$
                Apply the above result by setting $y$ to $x^2$, then we have
                $$\begin{aligned}
                    \int_{-1}^1(1-x^2)^n\,\mathrm{d}x
                    &\geq\int_{-1/\sqrt{n}}^{1/\sqrt{n}}(1-x^2)^n\,\mathrm{d}x\\
                    &\geq\int_{-1/\sqrt{n}}^{1/\sqrt{n}}1-nx^2\,\mathrm{d}x
                    =\frac{4}{3\sqrt{n}}.
                \end{aligned}$$
                Hence
                $$c_n\leq \frac{3\sqrt{n}}{4}\leq\sqrt{n}.$$

                \item[Subclaim 2:] $\forall \delta>0$, $Q_n\rightarrow0$
                uniformly on $\{x:\delta\leq|x|\leq 1\}$.\\
                On $\delta\leq|x|\leq 1$,
                $$Q_n(x)=c_n(1-x^2)^n\leq \sqrt{n}(1-\delta^2)^n\rightarrow0,$$
                since $1-\delta^2<1$.
            \end{description}
        Now let's prove the claim.
        Let $x\in[0,1]$.
        $$\begin{aligned}
            P_n(x)-f(x)
            &=\int_0^1f(t)Q_n(x-t)\,\mathrm{d}t-f(x)\int_{-1}^1Q_n(s)\,\mathrm{d}s\\
            &=\int_{x-1}^xf(x-s)Q_n(s)\,\mathrm{d}s-\int_{-1}^1f(x)Q_n(s)\,\mathrm{d}s\\
            &=\int_{-1}^1f(x-s)Q_n(s)\,\mathrm{d}s-\int_{-1}^1f(x)Q_n(s)\,\mathrm{d}s.
        \end{aligned}$$
        since $f=0$ off of $[0,1]$ (so $f(x-s)$ vanishes for $s\notin[x-1,x]$).
        Thus
        $$
            P_n(x)-f(x)
            =\int_{-1}^1(f(x-s)-f(x))Q_n(s)\,\mathrm{d}x
            =\int_{-\delta}^\delta+
            \int_{-1}^{-\delta}+
            \int_{\delta}^1 \eqqcolon I_1+I_2+I_3
        $$
        For $I_1$, we have
        $$\begin{aligned}
            I_1&\leq
            \int_{-\delta}^\delta|f(x-s)-f(x)|\cdot|Q_n(s)|\,\mathrm{d}s\\
            &\leq\max_{|s|\leq\delta}|f(x-s)-f(x)|\int_{-1}^1Q_n(s)\,\mathrm{d}s.\\
            &=\max_{|s|\leq\delta}|f(x-s)-f(x)|
        \end{aligned}$$
        For $I_2$ and $I_3$, we have
        $$\begin{aligned}
            I_2+I_3
            &\leq2\cdot\max_{s\in\mathbb{R}}|f(x-s)-f(x)|\cdot \max_{\delta\leq
            |s|\leq1}|Q_n(s)|
            \leq 4\cdot \|f\|_{\mathcal{C}^0}\cdot \max_{\delta\leq|s|\leq1}|Q_n(s)|
        \end{aligned}$$
        Now, let $\epsilon>0$.
        Since $f$ is uniformly continuous,
        $\exists \delta>0$, s.t.\ 
        whenever $|s|\leq\delta$, $|f(x-s)-f(x)|<\epsilon/2$, $\forall
        x\in\mathbb{R}$.
        So $I_1<\epsilon/2$, $\forall n$.
        Also, by the subclaim, $\exists N$, s.t. $\forall n\geq N$,
        $\max_{\delta\leq|s|\leq1} |Q_n(s)|<\epsilon/(4\|f\|_{\mathcal{C}^0})$.
        So $\forall n\geq N$ and $x\in[0,1]$, 
        $$|P_n(x)-f(x)|\leq|I_1|+|I_2+I_3|<\epsilon.$$
    \end{description}
\end{proof}

\begin{defn}
    Let $E$ be a nonempty set.
    $$\mathcal{F}(E)\coloneqq\mathcal{F}(E;\mathbb{R})\coloneqq\{f:E\rightarrow\mathbb{R}\}.$$
\end{defn}
\begin{defn}
    A family $\mathcal{A}\subseteq\mathcal{F}(E)$ is an algebra if
    $\forall c\in\mathbb{R}$, and $f,g\in\mathcal{A}$,
    $$cf,\,f+g,\,fg\in\mathcal{A}.$$
\end{defn}
\begin{defn}
    Let $\mathcal{A}\subseteq \mathcal{F}(E)$.
    \begin{itemize}
        \item
        $\mathcal{A}$ separates points in $E$ if $\forall x_1,x_2\in E$ with $x_1\neq x_2$,
        $\exists f\in\mathcal{A}$, s.t. $f(x_1)\neq f(x_2)$.
        \item
        $\mathcal{A}$ is nonvanishing on $E$ if $\forall x\in E$, $\exists
        f\in\mathcal{A}$, s.t. $f(x)\neq0$.
    \end{itemize}
\end{defn}

\begin{eg}
    The set $P$ of polynomials on $\mathbb{R}$ is an algebra, which separates
    points in $\mathbb{R}$ and vanishes nowhere.
\end{eg}
\begin{eg}
    The set $P_{odd}$ is not an algebra and is not nonvanishing (because it is
    always $0$ at $x=0$).
    The set $P_{even}$ is an algebra, which is nonvaishing on $\mathbb{R}$ but
    it doesn't separate points.
\end{eg}

\begin{prop}
    Let $\mathcal{A}\subseteq\mathcal{F}(E)$ be an algebra that separates points
    and vanishes nowhere.
    Then $\forall x_1\neq x_2\in E$ and $\forall c_1,c_2\in\mathbb{R}$,
    $\exists f\in\mathcal{A}$, s.t. $f(x_1)=c_1$ and $f(x_2)=c_2$.
\end{prop}
\begin{proof}
    By definition, $\exists g,h,k\in\mathcal{A}$, s.t.
    $g(x_1)\neq g(x_2)$, $h(x_1)\neq0$, $k(x_2)\neq0$.
    Now let
    $$\begin{aligned}
        u&=(g-g(x_1))k=gk-g(x_1)k\in\mathcal{A},\\
        v&=(g-g(x_2))h=gh-g(x_2)h\in\mathcal{A}.\\
    \end{aligned}$$
    Then
    $u(x_1)=0, u(x_2)\neq 0, v(x_1)\neq0, v(x_2)=0$.
    Finally, let
    $$f=c_1\frac{v}{v(x_1)}+c_2\frac{u}{u(x_2)}\in\mathcal{A}.$$
\end{proof}

\begin{thm}[Full Stone Weierstress]
    Let $K$ be a compact set and $\mathcal{A}\subseteq\mathcal{C}^0(K)$ be an
    algebra that separtes points and vanishes nowhere.
    Then $\mathcal{A}$ is dense in $\mathcal{C}^0(K)$.
    In other words, $\forall f\in\mathcal{C}^0(K)$, there exists a sequence
    $\{f_n\}$ in $\mathcal{A}$, s.t. $f_n\rightarrow f$ uniformly.
\end{thm}
\begin{proof}
    Let's $\mathcal{C}=\bar{\mathcal{A}}$.
    Claim $\mathcal{C}$ is an algebra (HW).
    We now need to show that $\mathcal{C}=\mathcal{C}^0(K)$.
    \begin{description}
        \item[Claim 1:] If $f\in\mathcal{C}$, then so is $|f|$.\\
        Let $a\coloneqq\|f\|_{\mathcal{C}^0(K)}$.
        By baby Stone Weierstress (aka Weierstress preparation lemma), there
        exists a polynomial $P$ on $\mathbb{R}$ such that
        $|P(y)-|y||<\epsilon$ for any $y\in[-a,a]$.
        Define $g\coloneqq P\circ f$.
        Then $g(x)=\sum_{n=0}^N a_nf(x)^n$, where the $a_n$'s are the coeffients of
        $P$.
        Since $\mathcal{C}$ is an algebra, $g\in\mathcal{C}$.
        Furthermore, $\forall x\in K$, $f(x)\in[-a,a]$.
        So
        $$|g(x)-|f(x)||=|P(y)-|y||<\epsilon.$$
        Thus $\||f|-g\|_{\mathcal{C}^0(K)}\leq\epsilon$.
        Since $\epsilon$ was arbitrary, $|f|\in\bar{\mathcal{C}}=\mathcal{C}$.

        \item[Claim 2:] If $f_1,\ldots,f_n\in\mathcal{C}$, then so are
        $\max\{f_1,\ldots,f_n\}$ and $\min\{f_1,\ldots,f_N\}$.\\
        If $N=2$, this follows from Claim 1 and
        $$\max\{f,g\}=\frac{f+g}{2}+\frac{|f-g|}{2},\quad
        \min\{f,g\}=\frac{f+g}{2}-\frac{|f-g|}{2}.$$
        For larger $N$, by induction, we have
        $$\begin{aligned}
            \max\{f_1,\ldots,f_{N+1}\}&=\max\{\max\{f_1,\ldots,f_N\},f_{N+1}\},\\
            \min\{f_1,\ldots,f_{N+1}\}&=\min\{\min\{f_1,\ldots,f_N\},f_{N+1}\}.
        \end{aligned}$$

        \item[Claim 3:] 
        Let $f\in\mathcal{C}^0(K)$, $\epsilon>0$ and $x_0\in K$.
        Then $\exists g_{x_0}\in\mathcal{C}$, s.t. $g_{x_0}(x_0)=f(x_0)$ and
        $g_{x_0}(x)>f(x)-\epsilon$ for all $x\in K$ (approximate $f$ from not
        too far below).\\
        Let $x_1\in K$. Then $\exists h_{x_1}\in\mathcal{C}$, s.t.
        $h_{x_1}(x_0)=f(x_0)$ and $h_{x_1}(x_1)=f(x_1)$.
        For $y\in K$, define
        $$G_y\coloneqq \{x\in K:h_y(x)>f(x)-\epsilon\}.$$
        Then $G_y$ is open since $h_y$ is continuous.
        Also $y\in G_y$ since $h_y(y)=f(y)$.
        Thus, $\{G_y:y\in K\}$ is an open cover of $K$, so $\exists
        y_1,\ldots,y_N\in K$, s.t. $K\subseteq\cup_{n=1}^N G_{y_n}$.
        Now, let $g_{x_0}=\max\{h_{y_1},\ldots,h_{y_N}\}$.
        By Claim 2, $g_{x_0}\in\mathcal{C}$.
        Furthermore, $g_{x_0}(x_0)=f(x_0)$.
        Finally, for $x\in G_{y_n}$, $g_{x_0}(x)\geq h_{y_n}(x)>f(x)-\epsilon$.
        Thus, $g_{x_0}$ is the function we want.

        \item[Claim 4:] $\forall f\in\mathcal{C}^0(K)$ and $\forall \epsilon>0$,
        $\exists g\in\mathcal{C}$, s.t. $\forall x$,
        $f(x)-\epsilon<g(x)<f(x)+\epsilon$.\\
        Let $x_0\in K$.
        By Claim 3, $\exists g_{x_0}\in\mathcal{C}$, s.t. $g_{x_0}=f(x_0)$ and
        $g_{x_0}(x)>f(x)-\epsilon$, $\forall x\in K$.
        For $y\in K$, define
        $$H_y\coloneqq \{x\in K: g_y(x)<f(x)+\epsilon\}.$$
        Then $H_y$ is open since $g_y$ is continuous.
        Again, $y\in H_y$ since $g_y(y)=f(y)$.
        So $\{H_y:y\in K\}$ is an open cover of $K$.
        Then $\exists y_1,\ldots,y_N\in K$, s.t.
        $K\subseteq\cup_{n=1}^N H_{y_n}$.
        Finally, define
        $g=\min\{g_{y_1},\ldots,g_{y_n}\}$.
        Then $\forall x$, $g(x)>f(x)-\epsilon$.
        Also $\exists n$, s.t. $x\in H_{y_n}$.
        Then $g(x)\leq g_{y_n}(x)<f(x)+\epsilon$.
    \end{description}
\end{proof}

\begin{thm}[Picard's theorem]
    Let $t_0\in\mathbb{R}$ and $y_0\in\mathbb{R}^k$.
    Let $a, b\in\mathbb{R}$ and define 
    $$B\coloneqq\{y\in\mathbb{R}^k:|y-y_0|\leq b\},$$
    and
    $$R\coloneqq [t_0-a,t_0+a]\times B.$$
    Let $F:R\rightarrow\mathbb{R}^k$ be a bounded, continuous function and let
    $M\coloneqq \|F\|_{\mathcal{C}^0(R)}$.
    Assume that $\exists C\in\mathbb{R}$, s.t. $\forall t\in(t_0-a,t_0+a)$,
    $\forall u,y\in B$, $|F(t,u)-F(t,y)|\leq C|u-y|$.
    Then, $\exists! $ function
    $g:(t_0-\tilde{a},t_0+\tilde{a})\rightarrow B$, s.t.
    $g$ is differentiable and $g$ solves the initial value problem
    $$\left\{
        \begin{aligned}
            &g(t_0)&&=y_0\\
            &g'(t)&&=F(t,g(t)),\quad\forall t\in(t_0-\tilde{a},t_0+\tilde{a}).
        \end{aligned}
    \right.$$
    Here, $\tilde{a}=\min\{a,b/M\}$.
\end{thm}
    Warning: The $C$ in the assumption is crucial for this theorem.
    Consider $k=1$, $F(t,y)=y^{1/3}$ and the initial value problem
    $$\left\{
        \begin{aligned}
            &g(0)&&=0\\
            &g'(t)&&=(g(t))^{1/3}
        \end{aligned}
    \right.$$

    Here is one solution, $g(t)=0$ for all $t$.
    Here is another solution,
    \[
        g(t)=\left\{
            \begin{aligned}
                &ct^{3/2},\quad&&t\geq0,\\
                &0,&& t<0,\\
            \end{aligned}
        \right.
    \]
    where $c^2=8/27$.
    Actually, there are infinitely many distinct solutions.
\begin{proof}
    Observe that $g$ solves our initial value problem if and only if
    \begin{itemize}
        \item $g$ is continuous;
        \item $|g(t)-y_0|\leq b$, $\forall t\in I$.
        \item $g(t)=y_0+\int_{t_0}^t F(s,g(s))\,\mathrm{d}s$, $\forall t\in
        I\coloneqq[t_0-\tilde{a},t_0+\tilde{a}]$.
    \end{itemize}
    Here we are using the fact that differentiable functions are continuous and
    equal the integral of their derivative plus some constant.
    On the other hand, that condition of $g$ implies continuity of the integrand
    and then we can use the fundamental theorem of calculus.

    Consider the set
    $$\mathcal{M}\coloneqq\{g\in\mathcal{C}^0(I;\mathbb{R}^k):g(t)\in B,\forall t\in I;
    g(t_0)=y_0\}.$$
    Consider the map $\Phi:\mathcal{M}\rightarrow \mathcal{C}^0(I,\mathbb{R}^k)$,
    $$[\Phi(g)](t)\coloneqq y_0+\int_{t_0}^t F(s,g(s))\,\mathrm{d}s.$$
    By the fundamental theorem of calculus, $\Phi(g)$ is differentiable and
    hence it is continuous.

    Now, we want to show that there exists an unique fixed point $g$ of $\Phi$ in
    $\mathcal{M}$.
    The idea is to apply contraction mapping theorem.
    Observe that $\mathcal{M}$ is closed since it is the intersection of two
    closed sets (careful here).
    Also since $\mathcal{C}^0(I,\mathbb{R})$ is complete, $\mathcal{M}$ is
    complete.
    Since the function $g(t)=y_0, \forall t\in I$ is in $\mathcal{M}$,
    $\mathcal{M}\neq\emptyset$.
    Thus, by the contraction mapping theorem, any contraction on $\mathcal{M}$
    has a unique fixed point $p$.
    Now we want to show $\Phi$ is a contraction on $\mathcal{M}$.

    Compute 
    $$\begin{aligned}
        |\Phi(g)(t)-y_0|=\left|\int_{t_0}^tF(s,g(s))\,\mathrm{d}s\right|
        &\leq \left|\int_{t_0}^t |F(s,g(s))|\,\mathrm{d}s\right|\\
        &\leq \left|\int_{t_0}^t M\,\mathrm{d}s\right|=|t-t_0|M\leq\tilde{a} M\leq b.
    \end{aligned}$$

    Let $g_1,g_2\in\mathcal{M}$.
    $$\begin{aligned}
        |\Phi(g_1)(t)-\Phi(g_2)(t)|
        &=\left|\int_{t_0}^tF(s,g_1(s)) - F(s,g_2(s))\,\mathrm{d}s\right|\\
        &\leq \left|\int_{t_0}^t |F(s,g_1(s)) -F(s,g_2(s))|\,\mathrm{d}s\right|\\
        &\leq \left|\int_{t_0}^t C|g_1(s)-g_2(s)|\,\mathrm{d}s\right|\\
        &\leq |t-t_0|\cdot C\cdot \|g_1-g_2\|_{\mathcal{C}^0(I,\mathbb{R}^k)}\\
        &\leq \tilde{a} C\|g_1-g_2\|_{\mathcal{C}^0(I;\mathbb{R}^k)}
    \end{aligned}$$
    So $\Phi$ is a contraction if and only if $\tilde{a}C<1$.
    We have two approaches to fix this.
    \begin{description}
        \item[Approach 1.]
        By contraction mapping theorem and above computation, we can uniquely
        solve the initial value problem for a shorter time, say on
        $I_0\coloneqq[t_0-a_0,t_0+a_0]$, where $a_0\coloneqq\min\{a,b/M,1/(2C)\}$.
        Now define $t_{-1}\coloneqq t_0-a_0$ and $t_{1}\coloneqq t_0+a_0$.
        Look at the new initial value problem on $[t_{-1}-a_0,t_1+a_0]$.
        Define $y_{\pm1}\coloneqq g_0(t_{\pm1})$.
        The new initial value problem can be written as
        $$\left\{
            \begin{aligned}
                g_{\pm1}(t_{\pm1})&=y_{\pm1}\\
                g_{\pm1}'(t)&=F(t,g(t))\\
            \end{aligned}
        \right.$$
        We a find a unique solution on
        $$I_{\pm 1}\coloneqq [t_{\pm1}-b_1, t_{\pm1}+b_1],$$
        where $b_1=\min\{a-a_0,\frac{b-a_0M}{M},1/2C\}$.
        Furthermore, by uniqueness (in contraction mappinig theorem),
        $g_{\pm 1}$ equals $g$ on $I_{\pm1}\cap I_0$.
        Thus there exists an unique solution on 
        $$I_1\cup I_0\cup I_{-1}=[t_0-a_0-b_1,t_0+a_0+b_1].$$
        Let $a_1\coloneqq a_0+b_1$.
        Then $a_1\geq \{a,b/M,2/2C\}$.
        Iterate this process will finish the proof.

        \item[Approach 2.]
        Find a better (but equivalent) metric on $\mathcal{M}$.
        \begin{defn}
            Two metrics $d_1$ and $d_2$ are equivalent if there exists positive
            constants $c_1,c_2$, s.t. $\forall f,g$,
            $$c_1 d_2(f,g)\leq d_1(f,g)\leq c_2d_2(f,g).$$
        \end{defn}
        \begin{prop}[HW]
            Two equivalent metrics yield the same open and closed set, the same
            continuous functions, the same Cauchy and convergent sequences.
        \end{prop}
        Now define 
        $$d_{C}(f,g)\coloneqq \sup_{t\in I}e^{-2C|t-t_0|}|g(t)-f(t)|.$$
        Then
        $$e^{-2C\tilde{a}}\|f-g\|_{\mathcal{C}^0(I,\mathbb{R}^k)}\leq
        d_C(f,g)\leq
        \|f-g\|_{\mathcal{C}^0(I,\mathbb{R}^k)}.$$
        So the metrics are equivalent and
        $\mathcal{M}$ is complete with respect to $d_C$.
        Finally, for $t\in I$,
        $$\begin{aligned}
            e^{-2C|t-t_0|}\left|\Phi(g_1)(t)-\Phi(g_2)(t)\right|
            &=e^{-2C|t-t_0|}\left |\int_{t_0}^t
            F(s,g_1(s))-F(s,g_2(s))\,\mathrm{d}s\right|\\
            &\leq e^{-2C|t-t_0|}\left|
            \int_{t_0}^t
            Ce^{2C|s-t_0|}e^{-2C|s-t_0|}|g_1(s)-g_2(s)|\,\mathrm{d}s\right|\\
            &\leq e^{-2C|t-t_0|}\left| \int_{t_0}^t
            Ce^{2C|s-t_0|}d_C(g_1,g_2)\,\mathrm{d}s\right|\\
            &\leq e^{-2C|t-t_0|}\cdot C\cdot d_C(g_1,g_2)\cdot\frac{1}{2C}e^{2C|t-t_0|}\\
            &=\frac{1}{2}d_C(g_1,g_2)
        \end{aligned}$$
        So $\Phi$ is a contraction on $\mathcal{M}$ with respect to $d_C$ and
        hence has unique fixed point.
    \end{description}
\end{proof}

\section*{Analytic Functions}
\begin{defn}
    $\mathbb{C}$ is the complex field defined by
    $$\mathbb{C}\coloneqq\{x+iy:x,y\in\mathbb{R}\}.$$
    Define $\overline{x+iy}\coloneqq x-iy$.
    Then $(x+iy)(x-iy) = x^2+y^2$.
    Define $|x+iy|\coloneqq \sqrt{x^2+y^2}$
    and $d(z,w)=|z-w|$.
    Note that $\mathbb{C}$ is a complete metric space.
    Recall that for a field, if $x_1+iy_1,x_2+iy_2\in\mathbb{C}$, then
    \begin{itemize}
        \item $(x_1+iy_1)+(x_2+iy_2) = (x_1+x_2) + i(y_1+y_2)\in\mathbb{C}$.
        \item $(x_1+iy_1)\cdot(x_2+iy_2) =
        (x_1y_1-x_2y_2)+i(x_1y_2+x_2y_1)\in\mathbb{C}$.
        \item $-(x+iy) = -x-iy\in\mathbb{C}$.
        \item If $x_2+iy_2\neq 0$ (i.e. $x_2\neq 0$ or $y_2\neq 0$), then
        \[
                \frac{(x_1+iy_1)}{(x_2+iy_2)} =
                \frac{(x_1+iy_1)}{(x_2+iy_2)}\cdot\frac{(x_2-iy_2)}{(x_2-iy_2)}
                =\frac{(x_1x_2+y_1y_2)}{x_2^2+y_2^2}+i\frac{(-x_1y_2+x_2y_1)}{x_2^2+y_2^2}.
        \]
    \end{itemize}
\end{defn}
\begin{defn}
    If $\{c_n\}$ is a sequence of complex numbers, $\sum_{n=0}^\infty c_n$
    converges if the sequence $\{s_n\}$ of partial sums
    $s_n\coloneqq\sum_{n=0}^N c_n$ converges.
    Furthermore, by completeness, $\{s_n\}$ converges if and only if $\{s_n\}$
    is Cauchy, in which case, we say $\{s_n\}$ satisfies the Cauchy criterion,
    that $\forall \epsilon$, $\exists N$, s.t. $\forall n>m\geq N$,
    $$\left|\sum_{j=m+1}^n c_j\right|<\epsilon.$$
\end{defn}

\begin{defn}
    $\sum_{n=0}^\infty c_n$ converges absolutely if $\sum_{n=0}^\infty |c_n|$
    converges.
\end{defn}

\begin{prop}
    If $\sum_{n=0}^\infty c_n$ converges absolutely, then $\sum_{n=0}^\infty
    c_n$ converges.
\end{prop}
\begin{proof}
    Use the Cauchy criterion and triangle inequality.
\end{proof}

\begin{defn}
    Let $X$ be a nonempty set and $\{f_n\}$ be a sequence of functions mapping from $X$
    into $\mathbb{C}$.
    Say that $\sum_{n=1}^\infty f_n$ converges (uniformly) if the sequence
    $s_n\coloneqq\sum_{j=0}^n f_n$ converges (uniformly).
\end{defn}

\begin{prop}
    $\sum_{n=0}^\infty f_n$ converges uniformly if and only if
    $\sum_{n=0}^\infty f_n$ satisfies a uniform Cauchy criterion, which means
    $\forall \epsilon>0$, $\exists N$, s.t. $\forall n\geq m\geq N$ and $\forall
    x\in X$, 
    $$\left|\sum_{j=m}^n f_j(x)\right|<\epsilon.$$
\end{prop}
\begin{note}
    Usually, it is much easier to show the convergence a sequence by showing
    that it is Cauchy.
\end{note}

\begin{thm}[Weierstress M test]
    Let $\{M_n\}$ be a sequence in $[0,\infty)$.
    Let $f_n$ be a sequence of functions mapping from $X\neq\emptyset$ into
    $\mathbb{C}$.
    Assume that $\forall n\in \mathbb{N}$ and $x\in X$, $|f_n(x)|\leq M_n$.
    Then, if $\sum_{n=0}^\infty M_n$ converges, then $\sum_{n=0}^\infty f_n$ converges
    uniformly on $X$.
\end{thm}
\begin{proof}
    Show $\sum_{n=0}^\infty f_n$ satisfies the uniform Cauchy criterion:
    $$\left|\sum_{j=m}^n f_j(x)\right|
    \leq \sum_{j=m}^n\left|f_j(x)\right|
    \leq\sum_{j=m}^n M_j.$$
    Since $\sum_{n=0}^\infty M_n$ is Cauchy, we are done.
\end{proof}

\begin{thm}[Root test]
    Let $\{c_n\}$ be a complex sequence in $\mathbb{C}$.
    Let $L\coloneqq \lim\sup|c_n|^{1/n}$.
    If $L<1$, then $\sum c_n$ converges absolutely.
    If $L>1$, then $\sum c_n$ diverges (badly).
\end{thm}
\begin{recall}
    Let $\{c_n\}$ be a sequence in $\mathbb{R}$.
    Then
    $$\lim\sup s_n = \sup\{\text{subsequence limits of }\{s_n\}\}
    =\lim_{n\rightarrow\infty}\sup\{s_k:k\geq n\}.$$
\end{recall}
\begin{proof}
    If $L<1$, then
    $\frac{L+1}{2}>L$.
    So $\exists N$, s.t. $\forall n\geq N$, $|c_n|^{1/n}<\frac{L+1}{2}$.
    Thus $|c_n|<\left(\frac{L+1}{2}\right)^n$ for $n\geq N$ and
    $\sum\left(\frac{L+1}{2}\right)^n$ converges since $\frac{L+1}{2}<1$.
    By comparison, $\sum|c_n|$ converges.\\
    If $L>1$, define $L'\coloneqq \min\{\frac{L+1}{2},2\}$.
    Since $L'<L$, $|c_n|^{1/n}>L'$ infinitely often.
    Thus $|c_n|>(L')^n$ infinitely often and since $L'>1$, $c_n$ cannot converge
    to zero. So $\sum c_n$ diverges.
\end{proof}

\begin{defn}
    Let $\{c_n\}$ be a complex sequence.  The radius of convergence of the power
    series $\sum_{n=0}^\infty c_n(z-z_0)^n$
    is
    $$R\coloneqq \lim\inf |c_n|^{-1/n}=\left(\lim\sup |c_n|^{1/n}\right)^{-1}.$$
\end{defn}

\begin{thm}
    Let $R$ denote the radius of convergence of the complex power series
    $\sum_{n=0}^\infty c_n(z-z_0)^n$.
    Then $\forall R'<R$, $\sum_{n=0}^\infty c_n(z-z_0)^n$ converges absolutely uniformly on
    $\{z\in\mathbb{C}:|z-z_0|\leq R'\}$ and $\sum_{n=0}^\infty c_n(z-z_0)^n$
    diverges on $\{z\in\mathbb{C}:|z-z_0|>R\}$.
\end{thm}
\begin{proof}
    If $|z-z_0|\leq R'<R$,
    $$\lim\sup \left|c_n(z-z_0)^n\right|^{1/n}=\lim\sup |c_n|^{1/n}\cdot
    |z-z_0|\leq\frac{1}{R}\cdot R'.$$
    So eventually, $|c_n|\cdot|z-z_0|^n<\alpha^n$ for some $\alpha\in(R'/R,1)$.
    Outside $\{|z-z_0|\leq R\}$, root test shows divergence.
\end{proof}

\begin{lemma}
    The series 
    $$f^{(k)}(y)=\sum_{n=k}^\infty c_n n(n-1)\cdots (n-k+1)(x-a)^{n-k},$$
    has the same radius of convergence as $f(x)=\sum_{n=0}^\infty c_n(x-a)^n$.
\end{lemma}
\begin{proof}
    We want to show
    $$\lim_{n\rightarrow\infty}\left(n(n-1)\cdots(n-k+1)\right)^{1/n}=1.$$
    Since $\lim_{n\rightarrow\infty}\left(n(n-1)\cdots(n-k+1)\right)^{1/n}\geq1$,
    it sufficies to show that the $\lim\sup$ is $\leq 1$.
    But 
    $$\lim_{n\rightarrow\infty}\sup \left(n(n-1)\cdots(n-k+1)\right)^{1/n}\leq \lim_{n\rightarrow\infty}
    n^{k/n}=(\lim_{n\rightarrow\infty} n^{1/n})^k=1.$$
\end{proof}

\begin{thm}
    Let $R$ be the radius of convergence of the power series $\sum_{n=0}^\infty c_n (x-a)^n$.
    Assume $R>0$ and define $f(x)=\sum_{n=0}^\infty c_n(x-a)^n$ on $I\coloneqq (a-R,a+R)$.
    Then $f$ is indefinitely differentiable on $I$ and we can differentiate it term by term:
    $$f^{(k)}(y)=\sum_{n=k}^\infty c_n n(n-1)\cdots (n-k+1)(x-a)^{n-k}.$$
\end{thm}
\begin{proof}
    Let $g_N(x)=\sum_{n=0}^N c_n(x-a)^n$.
    Let $0<R'<R$.
    By the lemma and previous theorem, $\{g_N\}$ is a Cauchy sequence in $\mathcal{C}^k((a-R',a+R'))$.
    Thus, $\{g_N\}$ converges in $\mathcal{C}^k((a-R',a+R'))$.
    By uniqueness of limits, the limit must be $f$.
    Thus $f$ is differentiable to order $k$ on $(a-R',a+R')$ and 
    $$f^{(k)}(x)=\lim_{N\rightarrow\infty} g_N^{(k)}(x)=\lim_{N\rightarrow\infty}\sum_{n=0}^N c_n(n\cdots
    (n-k+1))(x-a)^{n-k}.$$
    Since $k$ is arbitrary and $R'$ is arbitrary, we are done.
\end{proof}

\begin{corollary}
    Under the hypotheses of the theorem, $f^{(k)}(a)=k!c_k$, i.e. $c_k=\frac{1}{k!}f^{(k)}(a)$.
\end{corollary}


\begin{thm}
    Let $I$ denote the set of all points $x$ at which $\sum_{n=0}^\infty c_n(x-a)^n$ converges.
    Then
    $$f(x)\coloneqq \sum_{n=0}^\infty c_n(x-a)^n$$
    is continuous on $I$.
\end{thm}
\begin{proof}
    First, we do some reductions:
    \begin{itemize}
        \item By replacing $f(x)$ with $f(x+a)$, we can assume $a=0$.
        \item Also, without loss of generality, we just need to prove for $0<R<\infty$.
        This is because $R=0$, then there is only one point at which the infinity sum converges.
        If $R=\infty$, then $I=\mathbb{R}$.
        So we can  replace $f(x)$ with $f(Rx)$ and assume $R=1$.
        \item It suffices to prove continuity at each interval and replacing $f(x)$ with $f(-x)$, it
        sufficies to prove the following theorem.
    \end{itemize}
\end{proof}

\begin{thm}
    If $\sum c_n$ converges, then $\sum c_n x^n$ converges $\forall |x|<1$ and $\lim_{x\rightarrow
    1^-}\sum c_n x^n = \sum c_n$.
\end{thm}
\begin{proof}
    Define $s_{-1}\coloneqq 0$, $s_n\coloneqq \sum_{j=0}^n c_j$, $s\coloneqq\sum_{j=0}^\infty c_j$.
    Define 
    $$f(x)\coloneqq \sum_{n=0}^{\infty} c_{n}x^n ,\quad x\in (-1,1].$$
    Then we can write
    $$\sum_{n=0}^m c_n x^n = \sum_{n=0}^m (s_n-s_{n-1})x^n
    =\sum_{j=0}^{m-1}s_j(x^j-x^{j+1})+s_mx^m.$$
    If $|x|<1$, $s_mx^m\rightarrow0$ as $x\rightarrow\infty$ since $s_m$ is bounded.
    Thus, for $|x|<1$,
    $$f(x)=\sum_{j=0}^\infty s_j(x^j - x^{j+1}) = (1-x)\sum_{j=0}^\infty s_j x^j.$$
    Since $\sum_{j=0}^\infty x^j = 1/(1-x)$, we can write
    $$\sum_{n=0}^\infty c_nx^n - \sum_{n=0}^\infty c_n = f(x)-f(1)=(1-x)\sum_{j=0}^\infty s_j x^j-
    s(1-x)\sum_{j=0}^\infty x^j=(1-x)\sum_{j=0}^\infty (s_j-s) x^j.$$
    Let $\epsilon>0$.
    Choose $N$, s.t. $\forall j\geq N$, $|s_j-s|<\epsilon$.
    Thus,
    $$\left|(1-x)\sum_{j=N}^\infty (s_i-s)x^j\right|
    <(1-x)\sum_{j=N}^\infty \epsilon |x|^j
    <\epsilon\cdot \frac{1-x}{1-|x|}<\epsilon,$$
    if $0\leq x\leq 1$.
    Since $\sum c_n$ converges, $\sum c_nx^n$ also converges.
    Furthermore, let $\delta\coloneqq\epsilon/\sum_{j=0}^{N-1}|s-s_j|$.
    Then if $1-\delta<x<1$,
    $$\left|(1-x)\sum_{j=0}^{N-1}(s-s_j)x^j\right|
    \leq (1-x)\sum_{j=0}^{N-1}|s-s_j|<\delta\cdot \sum_{j=0}^{N-1}|s-s_j|= \epsilon.$$
    Combining with the previous result, we obtain
    \[
        |f(x) - f(1)| = \left|(1-x)\sum_{j=0}^\infty (s_j-s)x^j\right| <
        2\epsilon.
    \]
\end{proof}

\begin{defn}
    The Cauchy product of the two series $\sum_{n=0}^\infty a_n$ and $\sum_{n=0}^\infty b_n$ is the
    series $\sum_{n=0}^\infty c_n$, where $c_n\coloneqq a_0b_n+a_1b_{n-1}+\cdots + a_nb_0$.
\end{defn}

\begin{thm}
    Assume $\sum a_n$ converges absolutely and $\sum b_n$ converges.
    Then their Cauchy product $\sum c_n$ converges.
    Furthermore, it converges to $\sum a_n \cdot \sum b_n$.
    If $\sum b_n$ converges absolutely as well, then $\sum c_n$ converges absolutely.
\end{thm}
\begin{proof}
    Some notations:
    $A_N\coloneqq \sum_{n=0}^N a_n$, $B_N\coloneqq \sum_{n=0}^N b_n$, $C_N\coloneqq \sum_{n=0}^N
    c_n$, $A\coloneqq \sum a_n$, $B\coloneqq \sum b_n$, $\bar{A}=\sum |a_n|$, $M\coloneqq \sup B_N$. 
    $$\begin{aligned}
        C_N&=c_0+\cdots + c_N\\
        &= a_0b_0 + (a_0b_1+a_1b_0) + \cdots + (a_0b_N+\cdots + a_Nb_0)\\
        &= a_0B_N + a_1B_{N-1} + \cdots + a_NB_0\\
        &= A_N B + a_0(B_N-B) + \cdots + a_N(B_0-B)\\
    \end{aligned}$$
    Define $\Gamma_N\coloneqq a_0(B_N-B) + \cdots + a_N(B_0-B)$.
    If we can show that $\lim_{N\rightarrow\infty}\Gamma_N=0$, it follows that $\lim_{n\rightarrow\infty}
    C_n = AB$.

    Let $\epsilon>0$.
    By absolute convergence of $\sum a_n$, $\exists N$, s.t. $\forall n\geq m\geq N_1$,  $\sum_{j=m}^n
    |a_j|<\epsilon/2M$.
    Since $B_n\rightarrow B$, $\exists N_2$, s.t. $\forall n>N_2$, $|B_n-B|<\epsilon/\bar{A}$.
    Finally, if $N\geq N_1+N_2$,
    $$\begin{aligned}
        |\Gamma_N|&\leq \sum_{j=0}^N |a_j|\cdot |B_{N-j}-B|\\
        &= \sum_{j=0}^{N_1} |a_j|\cdot |B_{N-j}-B| + \sum_{j=N_1+1}^{N} |a_j|\cdot |B_{N-j}-B|\\
        &<\sum_{j=0}^{N_1} |a_j|\cdot\frac{\epsilon}{\bar{A}} + \sum_{j= N_1+1}^N |a_j|\cdot 2M\\
        &<\bar{A}\cdot\frac{\epsilon}{\bar{A}} + \frac{\epsilon}{2M}\cdot 2M\\
        &=2\epsilon
    \end{aligned}$$
    Now we want to show that $\sum c_n$ converges absolutely if $\sum b_n$ converges absolutely.
    Note,
    $$\begin{aligned}
        \sum_{n=0}^N |c_n| &= |c_0|+\cdots+|c_N|\\
        &=|a_0b_0| + |a_0b_1+a_1b_0| + \cdots + |a_0b_N + \cdots + a_Nb_0|\\
        &\leq|a_0|\cdot |b_0| + \left(|a_0|\cdot |b_1|+|a_1|\cdot|b_0|\right)
        + \cdots + \left(|a_0|\cdot|b_N| + \cdots + |a_N|\cdot|b_0|\right)\\
    \end{aligned}$$
    Since $\sum |a_n|$ and $\sum |b_n|$ are both convergent, we are done.
\end{proof}

\begin{eg}
    Consider the Cauchy product of the two absolutely convergent power series
    $$\begin{aligned}
        &\sum_{n=0}^\infty a_n\quad\text{with}\quad a_n=\frac{1}{n!}x^n,\\
        &\sum_{n=0}^\infty b_n\quad\text{with}\quad b_n=\frac{1}{n!}y^n.\\
    \end{aligned}$$
    By definition, we have
    $$\begin{aligned}
        c_n &= a_0b_n+\cdots + a_nb_0 \\
        &= \sum_{j=0}^n a_j b_{n-j}\\
        &= \sum_{j=0}^n \frac{1}{j!}x^j \frac{1}{(n-j)!}y^{n-j}\\
        &= \frac{1}{n!}\sum_{j=0}^n\frac{n!}{j!(n-j)!}x^j y^{n-j}\\
        &= \frac{1}{n!}(x+y)^n
    \end{aligned}$$
    Upshot:
    $$\left( \sum_{n=0}^\infty \frac{1}{n!}x^n\right)
    \left(\sum_{n=0}^\infty \frac{1}{n!} y^n\right)
    =\sum_{n=0}^\infty \frac{1}{n!}(x+y)^n,$$
    i.e. $e^x e^y = e^{x+y}$.
\end{eg}

\section*{Unordered Series}
\begin{defn}
    Let $S$ be a set and $\{a_x\}_{s\in S}$ be a function from $S$ into $\mathbb{R}$.
    We say that the unordered series $\sum_{s\in S}a_s$ converges to $b\in\mathbb{R}$ if $\forall
    \epsilon>0$, $\exists$ a finite set $S_\epsilon\subseteq S$, s.t. $\forall$ finite set $S'$ with
    $S_\epsilon\subseteq S'\subseteq S$,
    \[\left| \sum_{s\in S'}a_s-b\right| < \epsilon.\]
\end{defn}
\begin{prop}
    An unordered series can have at most one sum.
\end{prop}

\begin{thm}
    The following are equivalent:
    \begin{enumerate}
        \item The unordered series $\sum_{s\in S}a_s$ converges.
        \item $\forall \epsilon>0$, $\exists$ a finite set $S_\epsilon\subseteq S$, s.t. $\forall$ finite
        set $S'\subseteq S\setminus S_\epsilon$, $\sum_{s\in S'}|a_s|<\epsilon$.
        \item $\sum_{s\in S}|a_s|$ converges absolutely.
        \item $\sup\{\sum_{s\in S'} |a_s| : S'\subseteq S \text{ is finite}\} < \infty$.
    \end{enumerate}
\end{thm}

\begin{proof}
    \begin{description}
        \item [$1\Rightarrow 2$:]
            Assume $\sum_{s\in S} a_s$ converges to $b$ and let $\epsilon>0$.
            Then $\exists$ a finite set $S_\epsilon\subseteq S$, s.t. $\forall$ finite set $S'$ with
            $S_\epsilon\subseteq S'\subseteq S$, $\left| \sum_{s\in S'} a_s -b\right|<\epsilon$.
            Let $S''\subseteq S\setminus S_\epsilon$ be a finite set.
            Now let $S_+''\coloneqq \{s\in S'':a_s>0\}$ and $S_-''\coloneqq \{s\in S'':a_s<0\}$.
            Then 
            \[\begin{aligned}
            \left|\sum_{s\in S''} |a_s|\right|
            &= \left|\sum_{s\in S_+''}a_s - \sum_{s\in S_-''}a_s\right|\\
            &= \left|\sum_{s\in S_+''\cup S_\epsilon}a_s - \sum_{s\in S_\epsilon}a_s - 
            \sum_{s\in S_-''\cup S_\epsilon}a_s + \sum_{s\in S\epsilon}a_s\right|\\
            &\leq \left| \sum_{s\in S''_+\cup S_\epsilon} a_s-b\right| + 2\left| \sum_{s\in S_\epsilon}a_s-b\right|
            +\left|\sum_{s\in S_\epsilon\cup S_-''} a_s-b\right|\\
            &< 4\epsilon.
            \end{aligned}\]

        \item [$2\Rightarrow 4$:]
            $\sup\{\sum_{s\in S'} |a_s|: S'\subseteq S \text{ is finite}\}\leq \epsilon + \sum_{s\in
            S_\epsilon}|a_s|<\infty$.

        \item [$4\Rightarrow 3$:]
            Let $B\coloneqq \sup \{\sum_{s\in S'}|a_s|:S'\subseteq S \text{ is finite}\}$.
            We want to show that $\sum_{s\in S}|a_s|=B$.
            Let $\epsilon>0$.
            By the definition of $\sup$, there exists a finite subset $S_\epsilon\subseteq S$ such that
            $\sum_{s\in S_\epsilon}|a_s| > B-\epsilon$.
            So if $S'$ is a finite set with $S_\epsilon\subseteq S'\subseteq S$,
            $$B-\epsilon<\sum_{s\in S_\epsilon}|a_s|\leq \sum_{s\in S'}|a_s|\leq B.$$
            Thus, by definition, the series $\sum_{s\in S}|a_s|$ converges.

        \item[$3\Rightarrow 2$:]
            The argument is similar as the argument for showing $1\Rightarrow 2$.

        \item[$3\Rightarrow 1$:]
            %Define $(a_s)_+\coloneqq \max \{a_s,0\}$ and $(a_s)_-\coloneqq \max\{-a_s,0\}$.
            %Thus, $a_s=(a_s)_+ - (a_s)_-$ and $|a_s| = (a_s)_+ + (a_s)_-$.
            %By the lemma below, if $\sum_{s\in S} |a_s| $ converges, then $\sum_{s\in S} (a_s)_+$ and
            %$\sum_{s\in S}(a_s)_-$ converges.
            %So $\sum_{s\in S}a_s = \sum_{s\in S} (a_s)_+ - (a_s)_-$ converges.
            %
            Suppose for contradiction that the unordered series $\sum_{s\in S} a_s$ does not converge.
            This means for all $b\in\mathbb{R}$, $\exists \epsilon>0$, s.t. $\forall$ finite
            $S_\epsilon\subseteq S$, $\exists$ a finite set $S'$ with $S_{\epsilon}\subseteq S'\subseteq
            S$, s.t.
            \[\left|\sum_{s\in S'}a_s - b\right|>\epsilon.\]
            This implies
            \[\sum_{s\in S'}a_s - b>\epsilon,\quad\text{or}\quad \sum_{s\in S'}a_s - b<-\epsilon,\]
            which further implies
            \[b-\epsilon>\sum_{s\in S'}a_s >b+\epsilon.\]
            Now, let's choose $b$ to be the limit of $\sum_{s\in S}|a_s|$.
            We have
    \end{description}
\end{proof}


\begin{lemma}
    If $\sum_{s\in S}|b_s|$ converges and $|a_s|\leq |b_s|$ for all $s$.
    Then $\sum_{s\in S}|a_s|$ converges.
\end{lemma}
\begin{proof}
    Let $\epsilon>0$.
    If $\sum_{s\in S}|b_s|$ converges, by the theorem, there exists a finite set $S_\epsilon\subseteq S$,
    s.t. $\forall$ finite set $S'\subseteq S\setminus S_{\epsilon}$, $\sum_{s\in S'}|b_s|<\epsilon$.
    Then $\forall$ finite set $S'\subseteq S\setminus S_{\epsilon}$, $\sum_{s\in S'}|a_s| \leq \sum_{s\in
    S'}|b_s|<\epsilon$.
    Then $\sum_{s\in S}|a_s|$ converges.
\end{proof}

\begin{corollary}
    The unordered series $\sum_{n\in\mathbb{N}}a_n$ converges if and only if $\sum_{n=1}^\infty a_n$
    converges absolutely.
\end{corollary}

\begin{prop}[HW]
    Show directly (without the theorem), that if $\lambda\in\mathbb{R}$ and $\sum_{s\in S}a_s$,
    $\sum_{s\in S}b_s$ converges, then
    $$\sum_{s\in S}\lambda a_s = \lambda \sum _{s\in S} a_s$$ and $$\sum_{s\in S}a_s+b_s=\sum_{s\in
    S}a_s+\sum_{s\in S}b_s$$
\end{prop}

\begin{prop}
    The unordered series $\sum_{(i,j)\in\mathbb{N}^2} a_{i,j}$ converges if and only if $\sum_{j=1}^\infty
    \left(\sum_{i=1}^\infty |a_{i,j}|\right)$ converges.
    In this case,
    \[\sum_{(i,j)\in\mathbb{N}^2} a_{i,j} = \sum_{j=1}^\infty \sum_{i=1}^\infty a_{i,j} = \sum_{i=1}^\infty
    \sum_{j=1}^\infty a_{i,j}.\]
\end{prop}
\begin{proof}
    Idea: $\Leftrightarrow$ follows directly from $\sum_{j=1}^\infty \sum_{i=1}^\infty |a_{i,j}| =
    \lim_{N\rightarrow\infty} \lim_{M\rightarrow\infty} \sum_{j=1}^N\sum_{i=1}^M |a_{i,j}|$.
    For the identity, consider the positive and negative parts separately.
\end{proof}

\begin{defn}
    Let $G\subseteq \mathbb{R}$ be an open set and let $f:G\rightarrow\mathbb{R}$.
    Say that $f$ is analytic on $G$ if $\forall a\in G$, $\exists \epsilon>0$, s.t.
    $f(x)=\sum_{n=0}^\infty \frac{1}{n!}f^{(n)}(a)(x-a)^n$, on $(a-\epsilon,a+\epsilon)$ (i.e. $f$ has a power
    series representation on $(a-\epsilon,a+\epsilon)$).
\end{defn}

\begin{thm}
    $f$ is analytic on the open set $G\subseteq\mathbb{R}$ if and only if $G$ can be written as a union of
    open intervals at which $f$ has a power series representation.
\end{thm}
\begin{note}
    Every open subset of $\mathbb{R}$ is a countable union of disjoint open intervals.
\end{note}

\begin{thm}
    Suppose that $f(x)=\sum_{n=0}^\infty c_nx^n$ on $\{|x|<R\}$, some $R>0$.
    If $|a|<R$, then $f$ has a power series representation centered at $a$ converging on $|x-a|<R-|a|$.
\end{thm}
\begin{proof}
    Notice that
    \[f(x)=\sum_{n=0}^\infty c_n (x-a+a)^n = \sum_{n=0}^\infty \sum_{j=0}^n c_n \binom{n}{j} a^{n-j}
    (x-a)^j.\]
    Observe that
    \[\sum_{n=0}^\infty \sum_{j=0}^n \left| c_n \binom{n}{j} a^{n-j} (x-a)^j\right|
    = \sum_{n=0}^\infty |c_n|(|x-a|+|a|)^n.\]
    Converges on $|x-a|+|a|<R$.
    So by proposition, we can switch the order of summation,
    \[f(x)=\sum_{j=0}^\infty \left(\sum_{n=j}^\infty c_n \binom{n}{j} a^{n-j}\right) (x-a)^j,\quad
    |x-a|<R-|a|,\]
    and $\sum_{n=j}^\infty c_n \binom{n}{j} a^{n-j}$ is the coefficient $c_j(a)$.
    In particular, the sum representing $c_j(a)$ converges absolutely $\forall |a|<R$ and
    \[c_j(a) = \frac{1}{j!}f^{(j)}(a).\]
\end{proof}

\begin{thm}
    If $f$ and $g$ are analytic on the open interval $I$, then so are $f+g$ and $f\cdot g$.
    In particular, if $f$ and $g$ both have power series representations centered at $a$ on $(a-R,a+R)$, then
    so do $f+g$ and $f\cdot g$.
\end{thm}

\begin{thm}
    If $f$ is analytic on the open interval $I$, $g$ is analytic on the open interval $J$, and
    $g(J)\subseteq I$.
    Then $f\circ g$ is analytic on $J$.
\end{thm}
\begin{proof}
    Let $a\in J$.
    By translating and adding a constant to $g$, we can assume $a=g(a)=0$.
    Now we expand
    \[ f(y) = \sum_{n=0}^\infty b_n y^n,\quad g(x)=\sum_{k=0}^\infty c_k x^k,\quad\text{on }|y|<\epsilon,
    |x|<\delta.\]
    Define $\bar{g}(x)\coloneqq \sum_{k=0}^\infty |c_k|x^k$.
    The $\bar{g}(x)$ is continuous on $|x|<\delta$.
    So by shrinking $\delta$ if needed, we may assume that $|\bar{g}(x)|<\epsilon$, $|x|<\delta$.
    $$f\circ g(x)=\sum_{n=0}^\infty b_n(g(x))^n.$$
    By previous theorem and induction, for each $n$,
    \[(g(x))^n = \sum_{k=0}^\infty a_k^{(n)} x^k,\quad (\bar{g}(x))^n=\sum_{k=0}^\infty
    \bar{a}_k^{(n)}x^k,\quad\text{on }|x|<\delta.\]
    Furthermore, $|a_k^{(n)}|\leq \bar{a}_k^{(n)}$.
    Therefore,
    \[f\circ g(x) = \sum_{n=0}^\infty b_n \sum_{k=0}^\infty a_k^{(n)}x^k.\]
    We want to switch the order of summation, but it requires absolute convergence.
    Now
    $$\sum_{n=0}^\infty \sum_{k=0}^\infty |b_na_k^{(n)}x^k| \leq \sum_{n=0}^\infty \sum_{k=0}^\infty
    |b_n|\cdot\bar{a}_k^{(n)}\cdot |x|^k
    = \sum_{n=0}^\infty |b_n|(\bar{g}(|x|))^n,$$
    which converges since $\bar{g}(|x|)<\epsilon$.
    So
    \[f\circ g(x)=\sum_{k=0}^\infty\left( \sum_{n=0}^\infty b_na_k^{(n)}\right) x^k.\]
\end{proof}

\begin{corollary}
    If $f$ and $g$ are analytic on the open interval $I$ and $g\neq 0$ on $I$.
    Then $f/g$ is analytic on $I$.
\end{corollary}
\begin{proof}
    Since $1/x$ is analytic on $\mathbb{R}\setminus\{0\}$, by the previous theorem, $1/g$ is analytic.
    And so is $f\cdot 1/g=f/g$.
\end{proof}

\begin{thm}
    Let $E\subseteq \mathbb{R}$.
    Then $[x,y]\subseteq E$ for all  $x,y\in E$ if and only if $E=\emptyset$, $E$ is a singleton (i.e. $E$ is
    a single point), or $E$ is an interval.
\end{thm}
\begin{proof}
    \begin{description}
        \item[$\Leftarrow:$] This direction is trivial.
        \item[$\Rightarrow:$] Assume $\forall x,y\in E$, $[x,y]\subseteq E$.
        We may assume $E$ has at least $2$ points.
        Thus, $$\alpha\coloneqq \inf E<\sup E\eqqcolon \beta.$$
        Claim that $(\alpha,\beta)\subseteq E$.
        If so, we're done, since $E\subseteq [\alpha,\beta]$.\\
        Let $x\in(\alpha,\beta)$.
        Then $\exists y\in E$, s.t. $y<x$ and $\exists z\in E$, s.t. $z>x$.
        By hypothesis, $x\in[y,z]\subseteq E$.
    \end{description}
\end{proof}

\begin{thm}
    $E\subseteq \mathbb{R}$ is connected if and only if $E$ is empty, $E$ is singleton, or $E$ is an interval.
\end{thm}
    Notice, by the last theorem, the R.H.S. is equivalent to $[x,y]\subseteq E$ for all $x,y\in E$.
\begin{proof}
    \begin{description}
        \item [$\Rightarrow:$]
            It suffices to prove the contrapositive: If $E\neq \emptyset, E\neq\{x\}$ and $E$ is not an interval, then $\exists x,y\in E$ and $z$ with
            $x<z<y$ s.t. $z\notin E$.
            Then $E\cap (-\infty,z)$ and $E\cap (z,\infty)$ form a separation of $E$.
            So $E$ is not connected.
        \item [$\Leftarrow:$]
            Assume that $E = \emptyset $ or $E\neq\{x\}$ or $E$ is an interval and $E$ is not connected.
            Since a set that isn't connected must contain at least $2$ points, $E$ is an interval.
            Now we fix a separation $E=A\cup B$.
            By definition, $A, B\neq \emptyset$ and $\bar{A}\cap B = A\cap\bar{B}=\emptyset$.
            Fix $x\in A$ and $y\in B$.
            Since $A\cap B=\emptyset$, $x\neq y$.
            So we may assume $x<y$ (otherwise, just rename $A$ and $B$).
            Define $z\coloneqq \sup A\cap [x,y]$.
            Then, $z\geq x$ because $x\in A\cap [x,y]$ and $z\leq y$ because $y$ is an upper bound of $A$.
            So $z\in [x,y]\subseteq E = A\cup B$.
            Furthermore $z\in \overline{A\cap [x,y]} \subseteq \bar{A}$.
            Since $\bar{A}\cap B=\emptyset$, $z\notin B$.
            Then $z\in A$.
            Since $A\cap \bar{B}=\emptyset$, $z\notin \bar{B}$.
            So $\exists r>0$, s.t.\ $(z-r,z+r)\cap B=\emptyset$.
            But $y\geq z$ and $y\in B$, so $z+r\leq y$.
            So $z+r/2\in [z,y]\cap B^c\subseteq [x,y]\cap A$.
            But $z$ was an upper bound for $[x,y]\cap A$ and $z+r/2>z$.
            Contradiction!
    \end{description}
\end{proof}

\begin{thm}
    Let $I$ be an open interval and assume that $f$ and $g$ are analytic on $I$.
    Let $E\coloneqq \{ x\in I: f(x) = g(x) \}$.
    If $E$ has an accumulation point in $I$, then $E=I$, i.e.\ $f\equiv g$ on $I$.
\end{thm}
    Note, it is extremely important for $I$ to be an open interval and the assumption that $E$ has an accumulation point in $I$.
\begin{proof}
    By replacing $f$ with $f-g$, we may assume that $g\equiv 0$ on $I$.
    Since $f$ is continuous, $E$ is closed in $I$.
    Let $E'$ be a set of accumulation points of $E$.
    Therefore $E'\cap I\subseteq E$.
    Then $E'\cap I$ is closed in $I$.
    \begin{description}
        \item[Claim:] $E'\cap I$ is open.\\
            Let $a\in E'\cap I$.
            Then for some $\epsilon>0$, $f(x) = \sum_{n=0}^\infty \frac{1}{n!}f^{(n)}(a)(x-a)^n$, on $|x-a|<\epsilon$.
            Since $a\in E'\cap I\subseteq E$, $f(a)=f^{(0)}(a)=0$.
            Suppose $f(a)=f'(a)=\cdots = f^{(k)}(a) = 0$ and $f^{(k+1)}(a)\neq 0$ for some $k\geq 0$.
            By Taylor's theorem with remainder,
            \[f(x) = \frac{1}{(k+1)!}f^{(k+1)}(a)(x-a)^{k+1} + \frac{1}{(k+2)!}f^{(k+2)}(t_{x,a})(x-a)^{k+2},\]
            for some $t_{x,a}$ between $a$ and $x$.
            By continuity of $f^{(k+2)}$, there exists $\delta>0$, s.t.\ $\forall |x-a|<\delta$ and $|t-a|<\delta$, 
            \[ \frac{1}{(k+2)!}\left| f^{(k+2)}(t)\right|\cdot |x-a| < \frac{1}{2} \cdot \frac{1}{(k+1)!}\cdot \left|f^{(k+1)}(a)\right|.\]
            Thus for $0<|x-a|<\delta$,
            \[ |f(x)|\geq \frac{1}{2}\cdot \frac{1}{(k+1)!} \cdot \left| f^{(k+1)}(a)\right|\cdot |x-a|^{k+1}\neq 0.\]
            So $a\notin E'$. Contradiction.
            So $f^{(n)}(a) = 0$ for all $n$.
            So $f\equiv 0$ on $(a-\epsilon, a+\epsilon)$.
            So $(a-\epsilon,a+\epsilon)\subseteq E'\cap I$.
            Since $a$ was arbitrary, $E'\cap I$ is open.
    \end{description}
    Then we conclude that $E'\cap I$ is closed and open.
    Since $I$ is connected ($I$ is an open interval), $E'\cap I=\emptyset$ or $E'\cap I = I$.
    In the latter case, $E=I$.
\end{proof}

\subsection*{The exponential function}
\begin{property}
    Let
    $E(z)\coloneqq \sum_{n=0}^\infty \frac{1}{n!} z^n$, $z\in \mathbb{C}$.
    By root test, its radius of convergence is $\infty$ and it has the following properties:
    \begin{enumerate}
        \item $E(z)\cdot E(w) = E(z+w)$ for every $z,w\in\mathbb{C}$.
        \item $E(z)\neq0$ and $E(-z) = \frac{1}{E(z)}$ for every $z\in\mathbb{C}$.
        \item $E(x)>0$ for every $x\in \mathbb{R}$.
        \item $E'(x) = E(x)$ for every $x\in \mathbb{R}$.
    \end{enumerate}
\end{property}
\begin{proof}
    \begin{enumerate}
        \item By Cauchy products.
        \item $E(z)E(-z) = E(0) = 1$.
        \item For $x\geq 0$, $E(x)\geq 1$ by inspection.
            For $x< 0$, $E(x) = \frac{1}{E(-x)} > 0$.
        \item Differentiate term by term.
    \end{enumerate}
\end{proof}

\begin{defn}
    $e\coloneqq E(1)$.
\end{defn}

\begin{prop}
    $E(x) = e^x$, $\forall x\in\mathbb{R}$.
\end{prop}
\begin{proof}
    For $x=n\in\mathbb{N}$, this follows from property 1 and induction.
    $E(n) = E(1+\cdots + 1) = (E(1))^n = e^n$.
    For $x = n\in \mathbb{Z}$, this follows from preceeding and property $2$.
    For $x=\frac{n}{m}$ with $n\in\mathbb{Z}$, $m\in\mathbb{N}$, we know that
    \[ E\left(\frac{n}{m}\right)^m = E(n)\]
    by property 1.
    So by property 2 and uniqueness of $m$-th roots,
    $E(n/m) = E(n)^{1/m} = e^{n/m}$.
    So the conclusion holds for all rationals.
    Finally, for $x\in\mathbb{R}$, $e^x$ is (by definition, since $e>1$) $\sup\{e^p:p\in\mathbb{Q}\text{ and }p\leq x\}$.
    So the general case of proposition follows from continuity of $E$.
\end{proof}

\begin{thm}
    \begin{enumerate}
        \item $e^x$ is continuous and differentiable on $\mathbb{R}$ and $\frac{\mathrm{d}}{\mathrm{d} x}e^x = e^x$.
        \item $e^x$ is positive and strictly increasing on $\mathbb{R}$.
        \item $\lim_{x\rightarrow\infty}e^x= \infty$ and $\lim_{x\rightarrow-\infty}e^x = 0$.
        \item For fixed $n$, $\lim_{x\rightarrow\infty}e^x x^{-n} = \infty$ and $\lim_{x\rightarrow \infty}e^{-x}x^n = 0$.
    \end{enumerate}
\end{thm}
\begin{proof}
    \begin{enumerate}
        \item Proved by the 4th statement in the proposition.
        \item Combine the 3th statement in the proposition and the fact that $E(x) =e^x$, $e^x>0$.
            Also, since $E'(x) = E(x)>0$, so $e^x$ is strictly increasing.
        \item It is clear that $\lim_{x\rightarrow\infty} e^x = \lim_{x\rightarrow\infty}E(x) =  \infty$.
            Also by inspection,
            \[\lim_{x\rightarrow-\infty} e^x = \lim_{x\rightarrow\infty}e^{-x} = \lim_{x\rightarrow\infty}\frac{1}{e^x} = 0.\]
        \item For $x\geq 0$, $e^x\geq \frac{1}{(n+1)!} x^{n+1}$.
            So $e^x x^{-n}\geq\frac{1}{(n+1)!}x\rightarrow\infty$.
            On the other hand, $\lim_{x\rightarrow\infty} e^{-x}x^n = \lim_{x\rightarrow\infty}\frac{1}{e^x x^{-n}}= 0 $.
    \end{enumerate}
\end{proof}

\begin{lemma}
    Let $I$ and $J$ be intervals.
    Let $f: I\rightarrow J$ be a continuous bijection onto $J$.
    Then $f$ has a continuous inverse.
    Furthermore, if $y\in\text{int}(J)$ and $f$ is differentiable at $f^{-1}(y)$ with $f'(f^{-1}(y))\neq 0$.
    Then $f^{-1}$ is differentiable at $y$ and 
    \[(f^{-1})'(y) = \frac{1}{f'(f^{-1}(y))}.\]
\end{lemma}
\begin{note}
    The conclusion is not true if $I$ is an open interval and $J$ is a subset of an arbitrary metric space.
    For example, consider $I=\mathbb{R}$ and $J=8$.
    Map $\mathbb{R}$ into a bounded open interval (s.t.\ $\arctan$).
\end{note}
\begin{Remarks}
    If $K$ is compact and $f:K\rightarrow J$ is a continuous bijection, then $f^{-1}$ is continuous.
\end{Remarks}

\begin{proof}
    $f$ is either strictly increasing or strictly decreasing.
    Without loss of generality, we may assume $f$ is strictly increasing.
    Let $g\coloneqq f^{-1}$.
    Then $g$ is strictly increasing.
    \begin{description}
        \item[Claim 1:] $g$ is continuous.\\
            Let $y\in J$.
            Then both $g(y^+) = \lim_{t\rightarrow y^+}g(t)$ and $g(y^-) = \lim_{t\rightarrow y^-}g(t)$ exists (modify as
            needed if $y$ is an end point).
            Furthermore, we know $g(y^-) \leq g(y) \leq g(y^+)$.
            If $g(y^-) = g(y) = g(y^+)$, then $g$ is continuous at $g$.
            Let's suppose $g(y^-) < g(y)$.
            Because $g$ is increasing, then $g(t)\leq g(y^-)$ for all $t<y$ and $g(t)\geq g(y)$ for all $t\geq y$.
            So $g$ omits in $(g(y^-),g(y))\subseteq I$.
            But $I$ is the domain of $f$, which is the range of $g$.
            Contradiction.
            So $g(y^-)=g(y)$. Similarly, $g(y^+) = g(y)$.
        \item[Clam 2:] If $y\in \text{int}(J)$ and $f$ is differentiable at $f^{-1}(y)$ with $f'(f^{-1}(y))\neq0$.
            Then $f^{-1}$ is differentiable at $y$.\\
            We want to evaluate
            \[\lim_{h\rightarrow 0}\frac{g(y+h)-g(y)}{y+h-y} = \lim_{h\rightarrow0}\frac{g(y+h)-g(y)}{f(g(y+h)) -
            f(g(y))}.\]
            Define 
            \[\varphi(t)=\left\{
                    \begin{aligned}
                        &\frac{t-g(y)}{f(t)-f(g(y))},\qquad&& t\neq g(y),\\
                        &\frac{1}{f'(g(y))},&& t=g(y).\\
                    \end{aligned}
            \right.\]
            Then $\frac{1}{\varphi}$ is continuous on $I$ and $\frac{1}{\varphi(g(y))} = f'(g(y))\neq 0.$ 
            So $\varphi$ is continuous on a nerighbourhood of $g(y)$.
            So $\varphi\circ g$ is continuous at $y$, i.e.
            \[\lim_{h\rightarrow0}\frac{g(y+h)-g(y)}{f(g(y+h)) - f(g(y))} = \lim_{h\rightarrow 0}\varphi(g(y+h)) = \varphi(g(y)) = \frac{1}{f'(g(y))}.\]
    \end{description}
\end{proof}

\begin{defn}
    Since $E:x\mapsto e^x$ is continuous, differentiable, strictly increasing and maps $\mathbb{R}$ onto
    $(0,\infty)$. Thus it has an inverse, which we call $\log$, that is continuous, differentiable, strictly increasing
    and maps $(0,\infty)$ onto $\mathbb{R}$.
\end{defn}
\begin{property}
    \begin{enumerate}
        \item $e^{\log y} = y$ for all $y>0$ and $\log(e^x) = x$ for all $x\in \mathbb{R}$.
        \item $\frac{\mathrm{d}}{\mathrm{d}x}\log y = \frac{1}{y}$, $\forall y>0$.
        \item $\log(1) = 0$.
        \item $\log(y) = \int_1^y 1/s\,\mathrm{d}x$, $\forall y>0$.
        \item $\log(uv) = \log u + \log v$, $\forall u,v>0$.
        \item $\log(y)\rightarrow +\infty$ as $y\rightarrow +\infty$.
            $\log y\rightarrow-\infty$ as $y\rightarrow0$.
        \item For $u>0$ and $\alpha\in\mathbb{R}$, $\log(u^\alpha) = \alpha \log(u)$ and $u^\alpha = e^{\alpha\log u}$.
    \end{enumerate}
\end{property}
\begin{proof}
    \begin{enumerate}
        \item Done.
        \item By the lemma, $\log$ is differentiable and $\frac{\mathrm{d}}{\mathrm{d}y} \log(y) = \frac{1}{E'(\log
            (y))} = \frac{1}{E(\log(y))}=\frac{1}{y}$.
        \item Because $E(0) = 1$, done.
        \item By property two and three, done.
        \item $\log(uv) = \log(e^{\log u} \cdot e^{\log v}) = \log(e^{\log u+\log v}) = \log u + \log v$.
        \item Done.
        \item $\log(u^\alpha) = \log((e^{\log u})^\alpha) = \log(e^{\alpha \log u}) = \alpha \log u$.
    \end{enumerate}
\end{proof}

\begin{prop}
    For any $a>0$, $x\mapsto a^x$ is differentiable on $\mathbb{R}$ and
    \[\frac{\mathrm{d}}{\mathrm{d}x} a^x = (\log a)\cdot a^x.\]
\end{prop}
\begin{proof}
    $a^x = e^{x\log a}$. By the chain rule, done.
\end{proof}

\begin{prop}[HW]
    If $\alpha\neq -1$, then $x\mapsto x^\alpha$ has $x\mapsto \frac{x^{\alpha+1}}{\alpha+1}$ as an antiderivative.
\end{prop}

\begin{prop}
    $\forall \epsilon>0$, $0 = \lim_{x\rightarrow\infty} x^{-\epsilon} \log x= \lim_{x\rightarrow 0+} x^\epsilon \log x$.
\end{prop}
This is saying $\log$ goes to infinity very slow.
\begin{proof}
    Notice that
    \[\lim_{x\rightarrow 0+}x^\epsilon \log x = \lim_{y\rightarrow \infty}\left(\frac{1}{y}\right)^\epsilon
    \log\left(\frac{1}{y}\right) = -\lim_{y\rightarrow \infty}y^{-\epsilon}\log y.\]
    So it suffices to prove the first equation.
    \[\lim_{x\rightarrow \infty}x^{-\epsilon}\log x = \lim_{t\rightarrow \infty}(e^t)^{-\epsilon} \log(e^t) =
    \lim_{t\rightarrow\infty} e^{-\epsilon t}t = 0.\]
\end{proof}

\section*{Trigonometric Functions}
\begin{defn}
    Define
    \[C(z)\coloneqq \sum_{n=0}^\infty \frac{(-1)^n}{(2n)!} z^{2n}\quad\text{and}\quad S(z)\coloneqq \sum_{n=0}^\infty
    \frac{(-1)^n}{(2n+1)!}z^{2n+1},\]
    where $z\in\mathbb{C}$ (radius of convergence is $+\infty$).
\end{defn}
\begin{defn}
    Let $z\coloneqq x+iy$ where $x,y\in\mathbb{R}$. 
    Recall that $\bar{z} = x-iy$.
    Define $\text{Re}(z)\coloneqq x$ and $\text{Im}(z)\coloneqq y$.
    Then 
    \begin{itemize}
        \item $\text{Re}(z) = (z+\bar{z})/2$,
        \item $\text{Im}(z) = (z-\bar{z})/(2i)$,
        \item $\overline{z\cdot w} = \bar{z}\cdot\bar{w}$,
        \item $\overline{z+w} = \bar{z} + \bar{w}$.
    \end{itemize}
\end{defn}

\begin{note}
    For all $x\in \mathbb{R}$
    \[\overline{E(ix)} = \overline{\sum_{n=0}^\infty \frac{1}{n!}(ix)^n} = \sum_{n=0}^\infty\overline{
    \frac{1}{n!}(ix)^n} = E(\overline{ix}) = E(-ix).\]
    By this, we have
    \[C(x) = \sum_{n=0}^\infty \frac{(ix)^{2n}}{(2n)!} = \text{Re}\left(E(ix)\right) = \frac{1}{2}\left(E(ix) +
    E(-ix)\right).\]
    \[S(x) = \frac{1}{i} \sum_{n=0}^\infty \frac{(ix)^{2n+1}}{(2n+1)!} = \text{Im}\left(E(ix)\right) = \frac{1}{2i} (E(ix) -
    E(-ix)).\]
\end{note}

\begin{property}
    Trigonometric functions have the following properties:
    \begin{enumerate}
        \item $C(x)^2 + S(x)^2=1$, $\forall x\in\mathbb{R}$.
        \item $C(0) = 1$ and $S(0) = 0$.
        \item $C'(x) = -S(x)$, $S'(x) = C(x)$, $\forall x\in \mathbb{R}$.
    \end{enumerate}
\end{property}
\begin{proof}
    \begin{enumerate}
        \item $C(x)^2+S(x)^2 = \left|E(ix)\right|^2 = E(ix)\cdot\overline{E(ix)} = E(ix)E(-ix) = E(0) = 1$.
        \item Done.
        \item $C'(x) + iS'(x) = \frac{\mathrm{d}}{\mathrm{d}x}E(ix) = iE'(ix)=iE(ix) = iC(x) + i^2S(x)$.
            So we have $C'(x) = -S(x)$, $S'(x) = C(x)$.
    \end{enumerate}
\end{proof}

\begin{prop}
    $C$ has a positive zero.
\end{prop}
\begin{proof}
    Since $C(0) = 1\neq 0$ and since $C(x) = C(-x)$, it suffices to show $C$ has a zero.
    Suppose not.
    Since $C(0) > 0$ and $C$ is continuous, $C(x) > 0$, $\forall x\in\mathbb{R}$. Then $S$ is strictly increasing since
    $S'=C$.
    So for $0<x<y$, 
    \[S(x)(y-x) = \int_x^y S(x)\,\mathrm{d}t<\int_x^y S(t)\,\mathrm{d}t = C(x)-C(y).\]
    Since $C(x)\leq 1$ (because $C^2+S^2=1$), and $C(y)> 0$, so 
    $S(x)<\frac{1}{y-x}$, $\forall 0<x<y$.
    Fix $x$ and let $y\rightarrow \infty$, we see $S(x) = 0$.
    But $S$ strictly increasing and $S(0) = 0$, so $S(x) > 0$.
    Contradiction!
    So $C$ must has a zero.
\end{proof}

\begin{defn}
    Define $\pi\coloneqq 2 \inf\{x>0:C(x) = 0\}$.
    By continuity and $C(0)\neq 0$, $\pi > 0$ and the infimum is a minimum, i.e.\ $C(\pi/2) = 0$.
\end{defn}

\begin{thm}
    The following statements hold:
    \begin{enumerate}
        \item $C(k\pi) = (-1)^k$, $S(k\pi) = 0$, $C(k\pi + \pi/2) = 0$ and $S(k\pi + \pi/2) = (-1)^k$, $k\in\mathbb{Z}$.
        \item $E(z+2k \pi i) = E(z)$, $\forall k\in\mathbb{Z}$ and consequently $C$ and $S$ are periodic with period
            $2\pi$.
    \end{enumerate}
\end{thm}
\begin{proof}
    \begin{enumerate}
        \item Since $C(\pi/2) = 0$ and $C^2+S^2 = 1$, we see $S(\pi/2) = \pm 1$.
            By definition of $\pi$ and continuity of $C$ and $C(0)>0$, $S$ is increasing on $[0,\pi/2]$ and since $S(0) =
            0$, $S(\pi/2) = 1$.
            So $E(i\pi/2) = C(\pi/2) + iS(\pi/2) = 0 + i$.
            So $E(i k \pi/2) = i^k$ and all identities in $1$ hold.
        \item $E(z+2\pi i k) = E(z)E(2\pi i)^k = E(z)$ by part $1$.
    \end{enumerate}
\end{proof}

\begin{lemma}
    If $E(it) = 1$ and $t\in[0,2\pi)$, then $t = 0$.
\end{lemma}
\begin{proof}
    Suppose $t\in (0,2\pi)$ and $E(it) = 1$. By the proof of part 1 of the previous theorem, $S\neq 0$ on $(0,\pi/2]$.
    Since $S(\pi/2 - t) = S(\pi/2 + t)$ (Exercise) so $S>0$ on $(0,\pi)$.
    Since $S(t) = -S(-t) = -S(2\pi - t)$, $S<0$ on $(\pi, 2\pi)$.
    So $t=\pi$.
    But $E(i\pi) = -1$.
\end{proof}


\begin{thm}
    If $z\in\mathbb{C}$ with $|z| = 1$, then $\exists! t\in[0,2\pi)$, s.t. $E(it) = z$.
\end{thm}
\begin{proof}
    The above lemma proves the uniqueness of $z$.
    It remains to prove the existence.
    Let $z\in\mathbb{C}$ with $|z|=1$.
    Write $z=x + iy$, $x, y\in\mathbb{R}$.
    Thus $x^2+y^2 = 1$.
    \begin{description}
        \item[Case 1:] $x\geq 0, y\geq 0$.\\
            Then $0\leq x\leq 1$ and since $\cos 0 = 1$ and $\cos \pi/2 = 0$, by the intermediate value theorem,
            $\exists t\in[0,\pi/2]$, s.t.\ $\cos t= x$.
            Furthermore, $y=\sqrt{1-x^2} = \sqrt{1-\cos^2 t} = \sqrt{\sin^2 t} = |\sin t|$.
            Since $\sin$ is increasing on $[0,\pi/2]$ (because $\sin' = \cos$ is nonnegative) and $\sin(0) = 0$, $\sin
            t\geq 0$ and $|\sin t| = \sin t$.
        \item[Case 2:] $x\geq 0, y>0$.\\
            Then $\bar{z}$ is in the first quadrant. By Case 1, there exists $t\in(0,\pi/2]$, s.t. $e^{it} = \bar{z}$.
            Therefore, $z = e^{-it} = e^{i(2\pi-t)}$ and $2\pi-t\in[3\pi/2, 2\pi)$.
        \item[Case 3:] $x < 0$.\\
            Then by Case 1 and Case 2, $\exists t\in(-\pi/2, \pi/2)$, s.t.\ $-z = e^{it}$.
            Therefore $z = (-1)e^{it} = e^{i(\pi + t)}$ and $\pi+t\in(\pi/2,3\pi/2)$.
    \end{description}
\end{proof}

\begin{corollary}
    The circumference of the unit circle is $2\pi$.
\end{corollary}
\begin{proof}
    \[\text{Circumference} = \int_0^{2\pi} |E'(it)|\,\mathrm{d}t = \int_0^{2\pi} |E(it)|\,\mathrm{d}t =
    \int_0^{2\pi}C^2(t) + S^2(t)\,\mathrm{d}t = 2\pi.\]
\end{proof}

\begin{corollary}
    If $z\in\mathbb{C}$ with $z\neq 0$, then $\exists! t\in[0,2\pi)$, s.t.\ $z=|z|e^{it}$.
\end{corollary}

\begin{thm}[Algebraic Completeness of $\mathbb{C}$]
    Let $P(z) = a_0 + a_1 z + \cdots + a_n z^n$ be a complex polynomial with $a_n\neq 0$ and $n\geq 1$.
    Then there exists $z_0\in\mathbb{C}$ such that $P(z_0) = 0$.
\end{thm}
\begin{proof}
    Let $\mu = \inf_{z\in\mathbb{C}}\left|P(z)\right|$.
    We claim that $\mu$ is a minimum (it is achieved).
    Indeed, $|P(z)|\geq |a_n|\cdot|z|^n - \sum_{j=0}^{n-1} |a_j|\cdot |z|^j$.
    So $\exists R$, s.t. $\forall |z|>R$, $|P(z)| \geq \mu + 1$.
    Therefore by the continuity of $|P(z)|$, $\mu = \inf_{|z|\leq R} |P(z)| = \min_{|z|\leq R} | P(z)|$.
    Thus $\exists z_0\in\mathbb{C}$, s.t. $|P(z_0)| = \mu$.
    If $\mu = 0$, we are done.

    So now suppose $\mu > 0$.
    Define $Q(z) = \frac{P(z+z_0)}{P(z_0)}$.
    Then $Q$ is a polynomial.
    $Q(0) = 1$ and $|Q(z)| \geq 1$ for all $z$ (because $P(z_0)$ is the minimum).
    Thus
    \[Q(z) = 1+\sum_{j=k}^n b_j z^j,\quad\text{with } b_k\neq 0.\]
    By the previous theorem, $\exists \theta\in[0,2pi/k)$, s.t.\ $e^{ik\theta} = -\frac{|b_k|}{b_k}$.
    Thus for $r>0$, 
    \[\begin{aligned}
            \left|Q(re^{i\theta})\right| &= \left|1 + |b_k| r^k \cdot \frac{e^{ik\theta} b_k}{|b_k|} + \sum_{k+1}^n b_j r^j
            e^{ij\theta}\right|\\
            &= \left|1 - |b_k| r^k + \sum_{k+1}^n b_j r^j e^{ij\theta}\right|\\
            &\leq 1-|b_k|r^k + \sum_{j=k+1}^n |b_j|r^j
    \end{aligned}\]
    Notice that $|b_j|r^j \leq \frac{1}{2} |b_k|r^k$ for sufficiently small $r$. 
    So we further have
    \[
        \left|Q(re^{i\theta})\right| \leq 1-\frac{1}{2}|b_k| r^k < 1.
    \]
    Contradiction! Since $|Q(x)|\geq 1$ for all $x$.
    Tracing back, we see $\mu = 0$. So there exists $z_0\in\mathbb{C}$ such that $P(z_0) = 0$.
\end{proof}

\begin{corollary}
    Let $P(z) = a_0 + a_1 z + \cdots + a_n z^n$ be a complex polynomial with $a_n\neq 0$ and $n\geq 1$.
    There exists $z_1,\ldots, z_n\in\mathbb{C}$, s.t.
    \[P(z) = a_n (z-z_1)\cdots (z-z_n).\]
\end{corollary}
\begin{proof}
    By the theorem, there exists $z_n\in \mathbb{C}$ s.t. $P(z_n) = 0$.
    By long division algorithm, $P(z) = (z-z_n)Q(z) + \text{constant}$, where $Q$ is a polynomial with degree $n-1$.
    Evaluating both sides at $z=z_n$, we see that the constant is zero, i.e. $z-z_n|P$.
    Now repeat this procedure and we're done.
\end{proof}

\section*{Banach Spaces}
\begin{defn}
    $X$ is a real (or complex) vector space if $\forall x,y\in X$ and $\alpha, \beta\in\mathbb{R}$ (or $\mathbb{C}$),
    $\alpha x + \beta y\in X$ and some axioms hold.
\end{defn}

\begin{defn}
    A norm on $X$ is a function $\|\cdot\|_X: X\rightarrow \mathbb{R}$ satisfying 
    \begin{itemize}
        \item $\forall x\in X$, $\|x\|_X\geq 0$ and $\|x\|_X=0$ if and only if $x=0$.
        \item $\forall \alpha\in\mathbb{R}$ (or $\mathbb{C}$), $\forall x\in X$, $\|\alpha x\|_X =
            |\alpha|\cdot\|x\|_X$.
        \item $\forall x, y \in X$, $\|x+y\|_X\leq \|x\|_X + \|y\|_X$.
    \end{itemize}
\end{defn}

\begin{defn}
    The normed vector space $(X, \|\cdot\|)$ is a Banach space if $X$ is a complete metric space with respect to the
    distance $d(x, y) = \|x-y\|$.
\end{defn}
\begin{eg}
    $\mathbb{R}^k$ with euclidean metric is a Banach space.
    For any interval $I\subseteq \mathbb{R}$, $\mathcal{C}^k(I)$ is a Banach space.
    For any metric space $X$, $\mathcal{C}^0(X)$ is a Banach space.
\end{eg}

\begin{defn}
    Define 
    \[l^\infty \coloneqq l^\infty (\mathbb{N}) \coloneqq \left\{\text{bounded sequences }\{x_n\}_{x\in \mathbb{N}}\text{ in
    }\mathbb{R}\right\}.\]
    Define $\|\{x_n\}\|_{l^\infty} \coloneqq \sup_n |x_n|$.
\end{defn}

\begin{defn}
    For $1\leq p<\infty$, 
    \[
        l^p \coloneqq l^p(\mathbb{N}) \coloneqq \left\{\text{real sequences }\{x_n\}_{n\in\mathbb{N}}\text{ with } 
        \|\{x_n\}\|_{l^p} < \infty\right\},
    \]
    where
    \[
        \|\{x_n\}\|_{l^p}\coloneqq \left(\sum |x_n|^p\right)^{1/p}.
    \]
\end{defn}
\begin{note}
    In fact, $(l^\infty(\mathbb{N}), \|\cdot\|_{l^\infty}) = (\mathcal{C}^0(\mathbb{N}), \|\cdot
    \|_{\mathcal{C}^0(\mathbb{N})})$. 
    So we have already seen that $l^\infty$ is a Banach space.
\end{note}

\begin{thm}[Holders inequality]
    Let $1\leq p, q\leq \infty$ with $\frac{1}{p} + \frac{1}{q} = 1$ ($1/\infty = 0$).
    Let $\{a_n\}\in l^p$ and $\{b_n\}\in l^q$.
    Then $\{a_nb_n\}\in l^1$ and
    \[\|\{a_nb_n\}\|_{l^1} = \sum_{n=1}^\infty |a_nb_n| \leq \|\{a_n\}\|_{l^p}\cdot\|\{b_n\}\|_{l^q}.\]
\end{thm}
\begin{proof}
    If $\{a_n\} = \{0\}$  or $\{b_n\} = \{0\}$, the inequality holds trivially.
    Now suppose $p=\infty$ and $q = 1$ (the argument also works for $p=1$ and $q=\infty$).
    We have
    \[
        \sum_{n=1}^\infty |a_nb_n| \leq \sum_{n=1}^\infty \|\{a_n\}\|_{l^\infty}\cdot |b_n| =
        \|\{a_n\}\|_{l^\infty}\cdot\|\{b_n\}\|_{l^1}.
    \]
    Now consider $p,q\neq\infty$.
    Replacing $\{a_n\}$ with $\{a_n/\|\{a_n\}\|_{l^p}\}$ and $\{b_n\}$ with $\{b_n/\|\{b_n\}\|_{l^q}\}$
    if needed, we may assume $\|\{a_n\}\|_{l^p} = \|\{b_n\}\|_{l^q} = 1$.
    \begin{description}
        \item[Claim:] For $1<p,q<\infty$ and $x, y\geq 0$, $xy\leq x^p/p + y^q/q$.\\
            Define 
            \[
                f_y(x) = \frac{x^p}{p} + \frac{y^q}{q} - xy.
            \]
            Then
            $f_y'(x) = x^{p-1} - y$ and $f_y''(x) = (p-1)x^{p-2}\geq 0$.
            So $f_y$ has a global minimum at the zero of $f_y'$, i.e. $x=y^{1/(p-1)}$.
            Remember that $1/q = (p-1)/p$, so
            \[\begin{aligned}
                    f_y(y^{1/(p-1)}) &= \frac{y^{p/(p-1)}}{p} + \frac{y^q}{q} - y^{p/(p-1)}\\
                                     & = y^q\left(\frac{1}{p} + \frac{1}{q} -1\right)\\
                                     &= 0.
            \end{aligned}\]
    \end{description}
    Then we see
    \[\begin{aligned}
            \sum_{n=0}^\infty |a_nb_n| &\leq \sum_{n=0}^\infty \frac{|a_n|^p}{p} + \frac{|b_n|^q}{q}\\
                                       &=\frac{1}{p}\sum_{n=1}^\infty |a_n|^p + \frac{1}{q}\sum_{n=1}^\infty |b_n|^q\\
                                       &= \frac{1}{p} + \frac{1}{q}\\
                                       &= 1.\\
    \end{aligned}\]
\end{proof}

\begin{prop}
    Let $1\leq p, q\leq \infty$ with $\frac{1}{p} + \frac{1}{q} = 1$.
    Then for every real sequence $\{a_n\}$,
    \[\left(\sum_{n=1}^\infty |a_n|^p\right)^{1/p} = \sup_{\{b_n\}\in l^q,\ \|\{b_n\}\|_{l^q} = 1} \sum_{n=1}^\infty
    |a_nb_n|.\]
\end{prop}
\begin{proof}
    By Holders inequality, RHS $\leq$ LHS.
    If $\{a_n\}\in l^p$, take $b_n = |a_n|^{p-1}$.
    Then
    \[\begin{aligned}
            \|\{b_n\}\|_{l^q} = \left(\sum_{n=1}^\infty \left(|a_n|^{p-1}\right)^q\right)^{1/q} &= \left(\sum_{n=1}^\infty
            |a_n|^p\right)^{1/q} = \|\{a_n\}\|_{l^p}^{p/q}.
    \end{aligned}\]
    So $\{b_n\}\in l^q$.
    Divide $\{b_n\}$ by $\|\{b_n\}\|_{l^q}$ to make $\|\{b_n\}\|_{l^q=1}$ and
    we have
    \[
        \sum_{n=1}^\infty |a_nb_n| = \sum_{n=1}^\infty \frac{|a_n|^p}{\|\{a_n\}\|_{l^p}^{p/q}} =
        \frac{\|\{a_n\}\|_{l^p}^p}{\|\{a_n\}\|_{l^p}^{p/q}} = \|\{a_n\}\|_{l^p}.
    \]
    If $\{a_n\}\notin l^p$, take
    \[b_n^N=\left\{
            \begin{aligned}
                &|a_n|^{p-1}/\text{normalizing factor},\quad&& n\leq N,\\
                &0,&&n>N.\\
        \end{aligned}
    \right.\]
    Then
    \[
        \lim_{N\rightarrow\infty}\sum_{n=1}^N |a_nb_n^N| = \lim_{N\rightarrow\infty}\left(\sum_{n=1}^N
        |a_n|^p\right)^{1/p} = \infty.
    \]
\end{proof}

\begin{thm}[Triangle inequality]
    For $\{a_n\},\ \{b_n\}\in l^p$, $\{a_n+b_n\}\in l^p$ and 
    \[
        \|(a_n+b_n)\|_{l^p} \leq \|\{a_n\}\|_{l^p} + \|\{b_n\}\|_{l^p}.
    \]
    Consequently, $l^p$ is a normed vector space.
\end{thm}
\begin{proof}
    Choose $q$ such that $\frac{1}{q} + \frac{1}{p} = 1$.
    Then by the above proposition, we see
    \[
        \begin{aligned}
            \|\{a_n+b_n\}\|_{l^p} 
            &= \sup_{\{c_n\}\in l^q, \|\{c_n\}\|_q=1} \sum_{n=1}^\infty |c_n|\cdot |a_n+b_n|\\
            &\leq \sup_{\|\{c_n\}\|_q = 1} \left(\sum_{n=1}^\infty |c_n a_n| + \sum_{n=1}^\infty |c_nb_n|\right)\\
            &\leq \|\{a_n\}\|_{l^p} + \|\{b_n\}\|_{l^p}.
        \end{aligned}
    \]
\end{proof}

\begin{defn}
    $\sum x_n$ converges absolutely if $\sum \|x_n\|$ converges.
    $\sum x_n$ converges if the sequence $s_n\coloneqq x_1+\cdots + x_n$ of partial sums converges.
\end{defn}
\begin{eg}
    Define
    \[
        l_K^\infty = \{\text{real sequences }\{x_n\}, \text{s.t. }x_n=0, \text{for all except finitely many }n\},
    \]
    where $K$ stands for compact support.
    Let $x_n \coloneqq (0,\ldots,0,2^{-n},0,\ldots)$ where $2^{-n}$ appears at the $n$th position.
    Then $\sum x_n$ converges absolutely but it doesn't converge in $l_K^\infty$.
\end{eg}

\begin{thm}
    The normed vector space $(X,\, \|\cdot\|)$ is a Banach space if and only if every absolutely convergence series in
    $X$ converges.
\end{thm}
\begin{proof}
    \begin{description}
        \item[$\Rightarrow$:]
            If $X$ is a Banach space and $\sum x_n$ converges absolutely, then by traingle inequality, for $n\geq m$,
            \[
                \|s_n - s_{m-1}\|
                = \|a_m + \cdots + a_n\|
                \leq \sum_{j=m}^n \|a_j\|.
            \]
            So the sequence of partial sums is Cauchy.
            Since $X$ is complete, the sequnce of partial sums converge and thus $\sum x_n$ converges.
        \item[$\Leftarrow$:]
            Assume that every absolute convergence series in $X$ converges.
            Let $\{x_n\}$ be a Cauchy sequence in $X$.
            Recall that a Cauchy sequence with a convergence subsequence must converges (in any metric space).
            We need to prove that $\{x_n\}$ has a convergent subsequence.
            The idea is to use telescoping series.

            By Cauchyness, there exists a subsequence $\{x_{n_k}\}$ s.t.\ $\|x_{n_{k+1}} - x_{n_k}\| \leq 2^{-(k+1)}$,
            $\forall k\in \mathbb{N}$.
            Then $\sum x_{n_{k+1}} - x_{n_k}$ converges absolutely, so it converges by hypothesis.
            Therefore, 
            \[
                \lim_{k\rightarrow\infty} x_{n_k} = x_{n_1} + \sum_{k=1}^\infty (x_{n_{k+1}}-x_{n_k}),
            \]
            which implies $\{x_{n_k}\}$ is a convergent subsequence of $\{x_n\}$.
    \end{description}
\end{proof}

\begin{thm}
    $l^p$ is a Banach space for $1\leq p\leq \infty$.
\end{thm}
\begin{proof}
    We've already checked the case when $p=\infty$.
    Let $1\leq p<\infty$ and assume $\sum \|\bm{x}_n\|_{l^p} \leq \infty$, where $\bm{x}_n$ is a sequence for each $n$.
    If we can show $\sum \bm{x}_n$ converges, then by the above theorem, $l^p$ is a Banach space.
    For each $k\in\mathbb{N}$, the $k$-th coordinate satisfy
    \[
        \sum_n |x_{n,k}| \leq  \sum_n \left(\sum_k |x_{n,k}|^p\right)^{1/p} = \sum_n \|\bm{x}_n\|_{l^p} < \infty.
    \]
    So we can define a sequence $\{y_k\}$ by $y_k = \sum_n x_{n,k}$.
    We need to show $\{y_k\} \in l^p$ and $\sum \bm{x}_n = \{y_k\}$ with
    convergence in $l^p$.
    Choose $q$ with $\frac{1}{q} + \frac{1}{p} = 1$ and pick any $\{a_k\}\in l^q$ with $\|\{a_k\}\|_{l^q} = 1$.
    We see
    \[
        \sum_k |a_ky_k| = \sum_k \left| a_k \sum_n x_{n,k}\right| \leq \sum_k \sum_n |a_k x_{n,k}|
        \leq \sum_n \sum_k |a_k x_{n,k}|
        \leq \sum_n \|\bm{x}_n\|_{l^p}\|\bm{a}_k\|_{l^q}
        = \sum_n \|\bm{x}_n\|_{l^p}
        < \infty.
    \]
    Thus, $\|\{y_k\}\|_{l^p}<\infty$.
    Question.
    To show $\sum_n \bm{x}_n = \{y_n\}$ is similar.
    Show $\sum_{n=N}^\infty \bm{x}_n\rightarrow 0$ in $l^p$ as $N\rightarrow\infty$ using a similar argument.
\end{proof}

\subsection*{Bounded linear operators}
\begin{defn}
    If $X, Y$ are real vector spaces, a map $T: X\rightarrow Y$ is linear if $\forall \alpha_1, \alpha_2\in \mathbb{R}$,
    $\forall x_1, x_2\in X$, $T(\alpha_1 x_1 + \alpha_2 x_2) = \alpha_1 T(x_1) + \alpha_2 T(x_2)$.
\end{defn}

\begin{defn}
    Let $T: X^{\text{normed v.s.}}\rightarrow Y^{\text{normed v.s.}}$ be linear.
    We say $T$ is a bounded linear operator (or $T$ is bounded) if $\exists C$, s.t. $\forall x\in X$,
    $\|T(x)\|_Y\leq C\cdot\|x\|_X.$
\end{defn}

\begin{eg}
    $\frac{\mathrm{d}}{\mathrm{d}x}: \mathcal{C}^k(I) \rightarrow \mathcal{C}^{k-1}(I)$ is a bounded linear operator.
\end{eg}
\begin{eg}
    $X = \mathcal{C}^0([a,b])$ and $Y=\mathbb{R}$, $T(f) = \int_a^b f(x)\,\mathrm{d}x$ is a bounded linear operator.
\end{eg}

For convention, we write $T(x)$ as $Tx$.
\begin{thm}
    Let $X, Y$ be normed vector spaces and $T: X\rightarrow Y$ being a linear operator.
    Then the following are equivalent:
    \begin{enumerate}
        \item $T$ is a bounded linear opeartor;
        \item $T$ is uniformly continuous on $X$;
        \item $T$ is countinuous on $X$;
        \item $T$ is continuous at $0$.
    \end{enumerate}
\end{thm}
\begin{proof}
    \begin{description}
        \item[$1\Rightarrow 2:$] By 1, we have $\forall x,y$, $\|Tx - Ty\| = \|T(x-y)\| \leq C\|x-y\|$.
            So $T$ is Lipschitz, which implies $2$.
        \item[$2\Rightarrow 3:$] Done by definition.
        \item[$3\Rightarrow 4:$] Done by definition.
        \item[$4\Rightarrow 1:$] Assume $T$ is continuous at $0$.
            Then $\exists \delta > 0$, s.t. $\|x\|\leq \delta$ implies $\|Tx\| \leq 1$.
            Now if $x=0$, $\|Tx\| = 0$.
            For $x\neq 0$, since $\left\|\delta x/\|x\|\right\| = \delta$, so $\left\|T(\delta x/\|x\|)\right\|\leq 1$.
            By linearity and arithmetic, 
            $\|Tx\|\leq \|x\|/\delta$.
            So $T$ is a bounded linear operator ($C=1/\delta$).
    \end{description}
\end{proof}

\begin{defn}
    Define
    \[
        L(X,Y)\coloneqq \left\{\text{bounded linear operators from }X\text{ to }Y\right\}.
    \]
    Then $L(X,Y)$ is a vector space, since
    \[\begin{aligned}
            \|(\alpha T+\beta S) x\|_Y & = \|\alpha(Tx) + \beta(Sx)\|_Y\\
                                       &\leq |\alpha|\cdot\|Tx\| + |\beta|\cdot \|Sx\|\\
                                       &\leq (|\alpha| C_1 + |\beta| C_2)\|x\|.
    \end{aligned}\]
    Define norm
    \[
        \|T\|_{X\rightarrow Y} \coloneqq \sup_{0\neq x\in
        X}\frac{\|Tx\|_Y}{\|x\|_X}.
    \]
\end{defn}

\begin{prop} The following properties holds,
    \begin{enumerate}
        \item $\|T\| = \sup_{\|x\|=1} \|Tx\|$;
        \item $\|T\| = \min\left\{ C:\|Tx\|\leq C\|x\|,\ \forall x\in X\right\}$, in particular, $\forall x$,
            $\|Tx\|\leq \|T\|\cdot\|x\|$;
        \item $\|\cdot\|$ is a norm;
        \item If $Y$ is a Banach space, so is $L(X,Y)$;
        \item If $X, Y, Z$ are normed vector spaces and $T\in L(X,Y)$, $S\in L(Y,Z)$, then $S\circ T\in L(X,Z)$ and
            $\|S\circ T\| \leq \|S\|\cdot \|T\|$.
    \end{enumerate}
\end{prop}
\begin{proof}
    \begin{enumerate}
        \item By definition
            \[
                \|T\| = \sup_{0\neq x\in X}\frac{\|Tx\|}{\|x\|} = \sup_{0\neq x\in X}\|T\frac{x}{\|x\|}\| =
                \sup_{\|y\|=1}\|Ty\|.
            \]
        \item 
            For any $C<\|T\|$, by definition of $\|T\|$, $\exists x\neq 0$, s.t. $\|Tx\|/\|x\| > C$.
            So $\|Tx\| > C\|x\|$.
            Also check that $\|T\|\geq \|Tx\|/\|x\|$, which implies $\|Tx\|\leq \|T\|\cdot\|x\|$.
        \item Done.
        \item Assume that $\{T_n\}$ is a Cauchy sequence.
            Then $\forall x$, $\{T_nx\}$ is a Cauchy sequence (because $\|T_n x - T_m x\| \leq \|T_n - T_m\|\cdot
            \|x\|$).
            Since $Y$ is complete, $\{T_nx\}$ converges.
            Define $Tx\coloneqq \lim T_nx$.
            We need to show that $T\in L(X,Y)$.
            We see $T$ is linear by linearity of limits and the $T_n$'s.
            For the boundedness of $T$,
            we see $\|Tx\| = \lim \|T_n x\| \leq (\lim\sup \|T_n\|)\|x\|$.
            Note that $\{\|T_n\|\}$ is Cauchy in $\mathbb{R}$ since
            \[
                \|T_m\| - \|T_n\| \leq \|T_m-T_n\|
            \]
            and hence $\{\|T_n\|\}$ is convergent and bounded.
            So $T$ is bounded.
        \item
            Apply 2, we have
            \[
                \|S\circ T(x)\| \leq \|S\|\cdot \|Tx\| \leq \|S\|\cdot \|T\|\cdot \|x\|.
            \]
    \end{enumerate}
\end{proof}

\begin{defn}
    Define the following norms on $\mathbb{R}^n$:
    \begin{enumerate}
        \item $\|x\|_{l_n^\infty} \coloneqq \max_{1\leq i\leq n}|x_i|$
        \item $\|x\|_{l_n^p} \coloneqq \left(\sum_{i=1}^n |x_i|^p\right)^{1/p}$, $1\leq p\leq \infty$.
    \end{enumerate}
    A linear transformation $T:\mathbb{R}^n\rightarrow \mathbb{R}^m$ takes the form $Tx=Ax$, where $A=(a_{ij})$ is a
    $m\times n$ matrix.
\end{defn}

\begin{prop}
    In the above notation,
    \begin{enumerate}
        \item $\|T\|_{l_n^\infty\rightarrow l_m^\infty} = \max_{i=1,\ldots,m} \sum_{j=1}^n |a_{ij}|$;
        \item $\|T\|_{l_n^1\rightarrow l_m^1} = \max_{j=1,\ldots,n} \sum_{i=1}^m |a_{ij}|$.
    \end{enumerate}
\end{prop}
\begin{proof}
    \begin{enumerate}
        \item Define $C\coloneqq\max_{i=1,\ldots,m} \sum_{j=1}^n |a_{ij}|$.
            We need to first show $\|Tx\|_{l_m^\infty} \leq C\|x\|_{l_n^\infty}$, $\forall x$.
            This is true since
            \[
                \max_{i=1,\ldots,m} \left|(Tx)_i\right| = \max_{i=1,\ldots, m} \left|\sum_{j}a_{ij}x_j\right|
                \leq \max_{i=1,\ldots, m}\sum_j |a_{ij}|\cdot \|x\|_{l_n^\infty}.
            \]
            Now we need to show if $C'<C$, then $\exists x$, s.t.\ $\|Tx\|_{l_m^\infty} > C'\|x\|_{l_n^\infty}$.
            This is enough to find $x\neq 0$, s.t.\ $\|Tx\| = C\|x\|$.
            This is always possible in finite dimensional vector space (may not be possible in infinity dimensional
            vector space).
            Choose $i$ to maximize $\sum_j |a_{ij}|$.
            Take 
            \[
                x_j = \text{sign}(\alpha_{ij})\coloneqq 
                \left\{
                    \begin{aligned}
                        &1\qquad&&\text{if }a_{ij}\geq 0\\
                        &-1&&\text{if }a_{ij}<0.\\
                    \end{aligned}
                \right.
            \]
            Then $\|Tx\|_{l^\infty} = \sum_j |a_{ij}| = C\|x\|_{l^\infty}$.
        \item Homework.
    \end{enumerate}
\end{proof}
%midterm ends here

\section*{The open mapping and closed graph theorem}
\begin{defn}
    Let $X, Y$ be metric spaces.
    A function $f: X\rightarrow Y$ is an open map if $f(U)$ is open in $Y$ whenever $U$ is open.
\end{defn}

\begin{thm}[Open Mapping Theorem]
    Let $X$ and $Y$ be Banach spaces.
    Then a surjective map $T\in L(X,Y)$ is also an open map.
\end{thm}
\begin{proof}
    Assume $T\in L(X,Y)$ is surjective.
    Then, $Y=\cup_{n=1}^\infty T(B_n(0))$.
    By the Baire Category Theorem ($Y$ is complete), for some $n$, $\overline{T(B_n(0))}$ has nonempty interior.
    Since $\overline{T(B_n(0))} = n\overline{T(B_1(0))}$, $\overline{T(B_1(0))}$ has nonempty interior.
    Suppose $y_0 \in \text{int}(\overline{T(B_1(0))})$.
    Then $\exists r>0$, s.t.\ $B_r(y_0) \subseteq \overline{T(B_1(0))}$.
    \begin{description}
        \item[Claim 1: ]$B_{2r}(0)\subseteq B_r(y_0) - B_r(y_0)\coloneqq \{y-y':y, y'\in B_r(y_0)\}$\\
            If $\|x\| < 2r$, $y_0+x/2, y_0-x/2\in B_r(y_0)$.
            If $y, y'\in B_r(y_0)$, then $\|y-y'\| \leq \|y-y_0\| + \|y_0-y'\| < 2r$.
    \end{description}
    So $B_{2r}(0)\subseteq \overline{T(B_1(0))} - \overline{T(B_1(0))} = \overline{T(B_2(0))}$.
    So $B_r(0)\subseteq \overline{T(B_1(0))}$.
    \begin{description}
        \item[Claim 2: ]$B_{r/2}(0) \subseteq T(B_1(0))$.\\
            Let $y_1\in B_{r/2}(0)$.
            Then $\exists x_1\in B_{1/2}(0)$, s.t. $\|y_1 - Tx_1\|<r/4$.
            Let $y_2\coloneqq y_1-Tx_1$.
            In general, given $y_n \in B_{2^{-n}r}(0) \subseteq \overline{T(B_{2^{-n}}(0)}$,
            $\exists x_n\in B_{2^{-n}}(0)$, s.t. $\|y_n - Tx_n\| < 2^{-(n+1)}r$.
            Set $y_{n+1} = y_n - Tx_n$.
            Then $y_{n+1}\in B_{2^{-(n+1}r}(0)$.
            So we repeat.
            We obtain sequences $\{y_n\}$ in $Y$ and $\{x_n\}$ in $X$, s.t. $\|x_n\| < 2^{-n}$, $\forall n$ and $\|y_n -
            Tx_n\|<2^{-(n+1)}r$.
            Notice that 
            \[
                \|y_n - Tx_n\| = \|y_{n-1} - Tx_{n-1} - Tx_n\| = \cdots = \|y_1 - \sum_{j=1}^n Tx_j\| = \|y_1 -
                T\sum_{j=1}^n x_j\|.
            \]
            Now $\sum x_n$ converges absolutely (since $\|x_n\|<2^{-n}$).
            Since $X$ is complete, $\exists x\in X$, s.t. $\sum x_n = x$.
            Moreover, $\|x\|\leq \sum \|x_n\| < \sum 2^{-n} = 1$.
            So $x\in B_1(0)$.
            Finally,
            \[
                y_1 = \lim_{n\rightarrow\infty} T\sum_{j=1}^n x_j = T(\lim_{n\rightarrow\infty} \sum_{j=1}^n x_j) = T(x)
                \in T(B_1(x)).
            \]
    \end{description}
    Let $U$ be open and let $y\in T(U)$.
    Then $\exists x\in U$, s.t. $Tx=y$.
    Since $U$ is open, $\exists \epsilon > 0$, s.t. $B_\epsilon(x)\subseteq U$.
    Notice that
    \[
        T(B_{\epsilon}(0)) + Tx = T(B_{\epsilon}(0) + x) = T(B_\epsilon(x)) \subseteq T(U),
    \]
    and $Tx=y$.
    Finally, by Claim 2, 
    \[
        B_{\epsilon r/2}(y) = B_{\epsilon r/2}(0) + y \subseteq \epsilon T(B_1(0)) + y \subseteq T(U).
    \]
\end{proof}

\begin{corollary}
    If $X$ and $Y$ are Banach spaces and $T\in L(X,Y)$ is a bijection.
    Then $T^{-1}$ is also a bounded linear operator.
\end{corollary}
\begin{proof}
    It suffices to show that $T^{-1}$ is continuous.
    Let $U\subseteq X$ be an open set.
    Then
    \[
        \text{preImage}(T^{-1}) = T(U)
    \]
    is open.
    So $T^{-1}$ is continuous.
\end{proof}

\begin{thm}[Closed Graph Theorem]
    Let $X, Y$ be Banach spaces.
    The map $T:X\rightarrow Y$ is a bounded linear operator if and only if the graph
    \[
        \Gamma_T\coloneqq \{(x,Tx)\in X\times Y: x\in X\}
    \]
    is a closed linear subspace of $X\times Y$.
    Here $X\times Y$ is the Banach space with operations $(x_1,y_1) + (x_2, y_2) = (x_1+x_2, y_1+y_2)$ and 
    $\alpha(x,y) = (\alpha x, \alpha y)$, $\|(x,y)\| = \|x\|_X + \|y\|_Y$.
\end{thm}
\begin{proof}
    \begin{description}
        \item[$\Rightarrow: $]
            Homework.
        \item[$\Leftarrow: $]
            Assume that $\Gamma_T$ is a closed linear subspace.
            \begin{description}
                \item[Claim 1: ]$T$ is linear.\\
                    Let $x, x'\in X$ and $\alpha, \alpha'\in\mathbb{R}$, then
                    $(x, Tx), (x', Tx')\in \Gamma_T$.
                    So $(\alpha x + \alpha'x', \alpha Tx + \alpha'Tx')\in\Gamma_T$.
                    Thus bu definition of $\Gamma_T$, $\alpha Tx + \alpha' Tx' = T(\alpha x+\alpha'x')$.
                \item[Claim 2: ]$T$ is continuous.\\
                    We know $\Gamma_T$ is a Banach space.
                    We consider the projections $P_x:\Gamma_T\rightarrow X$ with $(x,y)\mapsto x$
                    and $P_y:\Gamma_T\rightarrow Y$ with $(x,y)\mapsto y$.
                    Then $P_x$ and $P_y$ are bounded, since $\|y\|, \|x\| \leq \|(x,y)\|$.
                    Furthermore, $P_x$ is a bijection.
                    So $P_x^{-1} : X\rightarrow \Gamma_T$,  $x\mapsto (x, Tx)$.
                    Finally, $T=P_y\circ P_x^{-1}$.
            \end{description}
    \end{description}
\end{proof}

\subsection*{Invertible linear opeartors and the van Neumann series}
\end{document}

